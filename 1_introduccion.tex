\chapter{Introducción}
\epigraph{ \textit{He llamado a este principio, por el cual cada pequeña variación, si útil, es preservada, con el término de Selección Natural}}{--- Charles Darwin, El origen de las especies}

Este capítulo pretende introducir al lector en el contexto general donde se enmarca el problema de sincronización de semáforos en el Corredor Garzón. Inicialmente se describen las motivaciones y el enfoque seguido para el desarrollo del proyecto. A continuación se especifican los objetivos propuestos, las limitaciones y su alcance. Para finalizar se delinea la estructura del documento presentando una breve descripción del contenido de cada capítulo. 

\section{Motivación y contexto}

En gran parte del mundo, el parque automotor está creciendo de forma sostenida desde hace varios años. Este crecimiento provoca serios problemas relacionados con la congestión del tráfico, afectando el desarrollo de las ciudades y la calidad de vida de las personas \citep{Cepal2003}. Las congestiones de tráfico producen una progresiva disminución en la velocidad media de circulación, así como también un aumento del consumo de combustible, impactando directamente en la contaminación atmosférica y sonora. Uruguay, en particular su capital Montevideo no escapa a este fenómeno global. Aunque la situación en Montevideo no sea tan crítica como en otras ciudades del mundo, las autoridades municipales han tomado medidas para solucionar este problema, implementando un Plan de Movilidad Urbana que pretende mejorar la eficiencia del transporte público \citep{PlanMovilidad}.

Uno de los puntos principales del Plan de Movilidad Urbana de la ciudad de Montevideo involucra la construcción de corredores urbanos de tránsito con carriles exclusivos para ómnibus. El primer corredor exclusivo implementado fue el Corredor Garzón, ubicado en la ciudad de Montevideo, con una extensión de 6.5 km, incluyendo 24 cruces semaforizados y vías exclusivas para ómnibus. Desde su inauguración en el año 2012 el Corredor Garzón ha recibido críticas por no cumplir uno de sus principales objetivos: el de agilizar el transporte público. La Intendenta de Montevideo Ana Olivera admitió que la duración de los viajes en el Corredor Garzón aumentó considerablemente. En palabras al Diario \cite{olivera2015} indicó: 


\begin{quote}\small
	``En el momento más crítico llegaron a haber 24 minutos más, lo medimos, porque hubo un momento que se confundió mucho o se partidizó la crítica y nosotros queríamos tener los datos objetivos. Y aquí el dato objetivo era que los ciudadanos de la zona y en particular los de Lezica, no era que se sentían perjudicados, era que habían sido perjudicados. Y nosotros lo asumimos"
\end{quote}

Los métodos para optimizar el tráfico se pueden dividir en dos categorías. Por un lado, los métodos enfocados en la modificación de las rutas, por ejemplo agregando nuevas vías de circulación, o ensanchando las existentes. Esta alternativa permite lograr mejoras en la circulación del tránsito, pero como punto negativo exigen un alto costo monetario \citep{litman2009transportation} y disponer de espacio físico para implementar las modificaciones viales. Por otro lado se encuentran los métodos orientados a influir en el comportamiento de los conductores, que incluyen las técnicas para configurar los semáforos y agregar señalizaciones de tránsito, entre otros \citep{mckenney2013distributed}. Estos métodos son muchas veces la única opción viable cuando no se dispone de espacio físico o el capital de inversión que exigen la modificación de las vías de tránsito. Por lo tanto, el estudio de estrategias para la sincronización eficiente de semáforos con el objetivo de mejorar la velocidad promedio de los viajes se presenta como un aporte interesante y necesario para el desarrollo ordenado de las ciudades. 

El problema de sincronización de semáforos es un problema de optimización NP-difícil \citep{yang1996model}, por lo que los métodos computacionales exactos son solo útiles en instancias de tamaño reducido. Por este motivo deben utilizarse métodos heurísticos y metaheurísticos para resolver instancias realistas del problema. Los algoritmos evolutivos han demostrado su utilidad en la resolución de problemas complejos, y resultan candidatos interesantes para resolver el problema de sincronización de semáforos. 

En el marco del proyecto, se pretende diseñar un algoritmo evolutivo capaz de mejorar la velocidad promedio de los vehículos en la zona del Corredor Garzón, modificando la configuración de sus semáforos, con el objetivo de aportar una solución eficiente e innovadora para el desarrollo de la ciudad y mejorar la calidad de vida de los ciudadanos de Montevideo. Las particularidades del Corredor Garzón lo convierten en un reto complejo desde el punto de vista de la investigación, como consecuencia de la extensión del tramo, de la cantidad de cruces, del número de semáforos, las reglas de exclusividad, los diferentes tipos de tráfico, entre otras.



\section{Objetivos}

Los objetivos que se plantearon al inicio del proyecto incluyeron:

\begin{enumerate}
	\item Estudio del problema del tráfico y la sincronización de semáforos.
	\item Relevamiento de información sobre trabajos relacionados en el ámbito de control de tráfico y sincronización de semáforos.
	\item Diseño e implementación de un algoritmo evolutivo multiobjetivo, capaz de resolver eficientemente el problema de sincronización de semáforos en la zona del Corredor Garzón.
	\item Creación de instancias realistas del problema, incluyendo un mapa y datos precisos sobre la configuración relativa a semáforos, tráfico y reglas de tránsito obtenidos de la realidad actual.	
	\item Aplicación de técnicas de computación de alto desempeño para mejorar el rendimiento computacional de la solución implementada.

\end{enumerate}

 
\section{Enfoque}

La metodología de resolución del problema de sincronización de semáforos en el Corredor Garzón propone como primer punto el modelado del problema. En este modelo se incluye: la creación de instancias realistas basadas en datos recabados de la realidad del Corredor Garzón, y el uso de un simulador de tráfico para analizar el comportamiento de variables útiles para el modelado, como son la velocidad promedio de ómnibus y de otros vehículos. Se plantea como un objetivo importante recabar datos precisos de la realidad, por lo tanto se incluyen tres fuentes para obtener y validar la información: datos recabados in-situ, reuniones con responsables y técnicos de la Intendencia de Montevideo, y finalmente información disponible públicamente.
Como segundo punto de la metodología se plantea el diseño y desarrollo de un algoritmo evolutivo multiobjetivo que basado en el modelo anterior y modificando la configuración de los semáforos, sea capaz de mejorar la velocidad promedio de ómnibus y de otros vehículos en la zona del Corredor Garzón. El tercer punto propone la aplicación de técnicas de alto desempeño para mejorar la eficiencia computacional de la solución, al considerarse que el problema y las instancias desarrolladas son complejas e insumirán un gran consumo de recursos computacionales. 



%%Desde un principio se intentó dotar al proyecto de una buena aproximación de la realidad. En tal sentido se realizaron reuniones con responsables y técnicos de la Intendencia de Montevideo, incluyendo al Ing. Juan Pablo Berta del Servicio de Ingeniería de Tránsito en agosto del  2014 y con el Ing. Daniel Muniz del departamento de Informática en setiembre de 2014, para conocer la situación del tráfico capitalino, aprender de su experiencia y obtener datos que fueran útiles para el proyecto. En este sentido se accedió a información del posicionamiento GPS y velocidad de los ómnibus para una semana en particular, lo que permitió procesarla para obtener la velocidad media de los ómnibus en la zona de Garzón. Además se obtuvo información sobre las técnicas y métodos utilizados en conteos vehiculares realizados en la ciudad de Montevideo en años anteriores.
%
%Buscando una aproximación aún más precisa se realizaron trabajos de campo para determinar la configuración de los semáforos, la densidad de tráfico y el tiempo del recorrido. El mapa y la frecuencia de ómnibus son de acceso público así como el simulador utilizado.
%
%Se creará un programa que implemente un algoritmo evolutivo multiobjetivo que utiliza un simulador de tráfico para obtener las métricas a optimizar. Se busca obtener una nueva configuración de semáforos que en las simulaciones se comporte mejor que la situación actual basándonos en la velocidad promedio de ómnibus y del resto de los vehículos.
%
%Dada la complejidad del problema el algoritmo se ejecutará en paralelo y se utilizará la plataforma Cluster Fing para poder acelerar el tiempo real de procesamiento. Además se creará un escenario alternativo con modificaciones de la realidad actual con el objetivo de mejorar las métricas.

\section{Limitaciones y alcance}

El alcance geográfico que se plantea comprende la zona del Corredor Garzón, que incluye toda la extensión del Corredor, y dos paralelas a cada lado del mismo. Se pretende desarrollar un modelo preciso de la realidad pero se tiene en cuenta que es usual que algunos elementos de la realidad sean eliminados o simplificados, por tanto se buscará que estas simplificaciones no afecten los resultados de la solución propuesta. El modelado incluye el relevamiento manual de datos en el Corredor Garzón. Debido a los errores intrínsecos que pueden ocurrir en la toma de datos, no se puede asegurar la exactitud de los mismos, por este motivo se realizarán verificaciones con el objetivo de minimizar el impacto que pudieran tener.




\section{Aportes}
Los aportes del proyecto incluyen: estudio del estado del arte y el marco teórico para el tratamiento del problema del tráfico vehicular, creación de instancias realistas del problema utilizando datos recabados in-situ, y el diseño e implementación de un algoritmo evolutivo multiobjetivo que resuelve el problema de sincronización de semáforos en el Corredor Garzón. Este algoritmo utiliza técnicas de computación de alto desempeño y permite mejorar la velocidad promedio de ómnibus y otros vehículos para distintas densidades de tráfico (baja, media y alta). Estos aportes se complementan con el desarrollo de un sitio web que se puede acceder en la siguiente dirección: \url{http://www.fing.edu.uy/inco/grupos/cecal/hpc/AECG}, los interesados pueden consultar información sobre el proyecto y sus resultados. Adicionalmente, para enriquecer el trabajo, se agrega la redacción de un artículo de síntesis en idioma inglés de 10 páginas, con el objetivo de tomarlo como base para la redacción de un articulo de investigación para presentar en una conferencia internacional. Finalizando, se destaca el diseño de un póster con la descripción del proyecto, que fue presentado en Ingeniería deMuestra 2014.



\section{Estructura del documento}
El capítulo 2 presenta los fundamentos teóricos necesarios para comprender el resto del trabajo. Comienza describiendo el problema del tráfico vehicular y algunas formas de atacarlo entre las que se encuentra la creación de planes de movilidad y la sincronización de semáforos. El capítulo incluye información sobre corredores urbanos de tránsito y en concreto sobre el Corredor Garzón, así como una descripción acerca de algoritmos evolutivos y simuladores de tráfico. Complementariamente, se ofrece una reseña de los principales trabajos relacionados, haciendo especial foco en algoritmos genéticos para la sincronización de semáforos.

El capítulo 3 explica la estrategia de resolución del problema de sincronización de semáforos implementada en este trabajo. Se presenta el modelado del problema que se basa en: la creación de un mapa digital de la zona del Corredor Garzón compatible con el simulador de tráfico SUMO y el trabajo de campo realizado para obtener datos de la realidad actual, que serán usados en las simulaciones. El capítulo detalla el diseño e implementación del algoritmo evolutivo y el modelo de paralelismo utilizado.

El capítulo 4 presenta el análisis experimental del algoritmo evolutivo, describe los escenarios utilizados en su evaluación y los resultados numéricos obtenidos. Se especifica el escenario que modela la situación actual del Corredor Garzón y el escenario alternativo que presenta modificaciones sobre el escenario anterior con el objetivo de mejorar la velocidad promedio de circulación. Se ofrece un análisis de los resultados obtenidos al variar la función de \emph{fitness} y se realiza un breve estudio de la eficiencia computacional del algoritmo evolutivo.

El capítulo 5 plantea las conclusiones finales del trabajo, tanto sobre la metodología utilizada como de los resultados obtenidos. Para finalizar, el capítulo describe las posibles lineas de investigación para desarrollar en el futuro.






