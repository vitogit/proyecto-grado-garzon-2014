\chapter{Introducción}
\epigraph{ \textit{He llamado a este principio, por el cual cada pequeña variación, si útil, es preservada, con el término de Selección Natural}}{--- Charles Darwin, El origen de las especies}

En esta sección se pretende introducir al lector en el contexto general donde se desarrolla este trabajo así como los objetivos buscados.

\section{Motivación y contexto}

Los algoritmos evolutivos han demostrado su utilidad en problemas complejos y particularmente uno de ellos es la sincronización de semáforos. Se busca desarrollar un algoritmo que logre resolver este problema con buenas métricas.
Por la flexibilidad inherente de este algoritmo es que no está destinado a resolver el problema en una zona en particular sino que se podría aplicar en forma general.

En este sentido se eligió la zona del corredor Garzón que presenta particularidades que la destacan y la hacen interesante desde el punto de vista de la investigación. Su complejidad viene dada por el largo del tramo, la cantidad de cruces y semáforos, las distintas reglas de tráfico (por ejemplo exclusividad del ómnibus, o distinción para doblar a la izquierda), circulación de tráfico vehicular y transporte público, calles no paralelas, entre otros.




\section{Objetivos}

Estos son los objetivos básicos que se plantearon al inicio del  proyecto:

\begin{itemize}
	\item Estudio del problema del tráfico y la sincronización de semáforos.
	\item Revelamiento de información sobre trabajos relacionados en este ámbito.
	\item Creación de un algoritmo evolutivo paralelo que resuelva el problema en la zona del corredor Garzón.
	\item Creación de un mapa y  configuración relativa a semáforos, tráfico y reglas de tránsito que sea precisa y obtenida de la realidad actual.	
	\item Aplicar técnicas de computación de alto desempeño para aumentar el rendimiento de la solución.

\end{itemize}

 
\section{Enfoque}

Desde un primer momento se intentó dotar al proyecto de una buena aproximación de la realidad, en tal sentido se realizaron reuniones con el Ing. Juan Pablo Berta del Servicio de Ingeniería de Tránsito de la Intendencia de Montevideo en Agosto del  2014 y con el Ing. Daniel Muniz del departamento de Informática de la Intendencia en Setiembre de 2014 para conocer la situación del tráfico capitalino, aprender de su experiencia y obtener datos que nos fueran útiles para el proyecto.


Buscando una aproximación aún más precisa se realizaron trabajos de campo para determinar la configuración de los semáforos, la densidad de tráfico y el tiempo del recorrido. El mapa y la frecuencia de ómnibus son de acceso público así como el simulador utilizado.

Se creará un programa que implemente un algoritmo evolutivo multiobjetivo que utiliza un simulador de tráfico para obtener las métricas a optimizar. Se busca obtener una nueva configuración de semáforos que en las simulaciones se comporte mejor que la situación actual basándonos en la velocidad promedio de ómnibus y del resto de los vehículos.

Dada la complejidad del problema el algoritmo será paralelo y se utilizara la plataforma Cluster fing para poder acelerar el tiempo real de procesamiento. Además se realizará un escenario alternativo con modificaciones de la realidad actual con el objetivo de mejorar las métricas.

\section{Limitaciones y alcance}

La zona modelada comprende todo el tramo de el Corredor Garzón y dos caminos paralelos a ambos lados, que dada la configuración de las calles las cuales corren en diagonal fue un proceso complejo.
El revelamiento de tráfico hecho in-situ fue realizado para un número determinado de calles que contiene a Garzón y cinco cruces representativos. Se busca una aproximación útil y no un estudio detallado sobre el tráfico en la zona.

Como lo que se pretende modelar es el tráfico vehicular y transporte público no se realiza una simulación de peatones.


\section{Aportes}

\begin{itemize}
	\item Se desarrollo un sitio web en la siguiente dirección: \url{http://www.fing.edu.uy/inco/grupos/cecal/hpc/AECG} donde los interesados podrán acceder para encontrar información sobre el proyecto y los resultados. 
	\item Se realizó un \emph{paper} en idioma inglés de 10 páginas con el objetivo de presentarlo en conferencias internacionales.
	\item El proyecto fue presentado en Ingeniería demuestra 2014, siendo bien recibido por el público. Constatando de primera mano que la problemática es real y llegando a la conclusión que los Ingenieros tienen las herramientas necesarias para solucionar problemas que afectan directamente a la sociedad.

\end{itemize}






\section{Estructura del documento}
En el siguiente capítulo se hace un repaso sobre fundamentos teóricos necesarios para comprender el resto del trabajo. Se da información sobre corredores y en concreto el Corredor Garzón, así como un repaso breve sobre algoritmos evolutivos y simuladores de tráfico. Además se muestran los trabajo relacionados enfocando en algoritmos genéticos para la sincronización de semáforos.

En el capítulo 3 se explica como se modela el problema y la arquitectura diseñada  para resolverlo. Luego se comenta el trabajo de campo realizado para obtener datos de la realidad utilizados en la solución. Además se detalla el algoritmo genético utilizado así como la biblioteca y herramientas usadas.

El capítulo 4 cuenta con la descripción de los escenarios, los resultados de la evaluación del algoritmo y las comparaciones realizadas. En este caso tenemos el escenario que representa la situación actual y también un escenario alternativo con modificaciones los cuales se evalúan con el algoritmo. Además se realizan pruebas para comprender como varía el algoritmo al modificar su función de fitness y se realiza un breve análisis de la eficiencia computacional del algoritmo.

El capítulo 5 da las conclusiones finales y el trabajo a futuro que se puede realizar.






