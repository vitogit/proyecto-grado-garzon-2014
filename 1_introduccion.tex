\chapter{Introducción}
En esta seccion pretende introducir al lector en el contexto en el que se desarrolla este trabajo asi como los objectivos buscados.

\subsubsection{Motivación y contexto}
Con  la  idea  de  agilizar  el  transporte  público  entre  los
barrios Colon y Sayago por un lado y la Ciudad Vieja, Pocitos
y Malvín por otro, barrios que son bastante lejanos el uno del
otro en la ciudad de Montevideo Uruguay, se realiza una obra
cuya  intención radica  en que la  clase obrera demorará  menos
tiempo en llegar a su destino.
Supuestamente se iban a acortar los tiempos del transporte
público de la hora que llevaba ir de Colón a Ciudad Vieja por
ejemplo  a  aproximadamente  25  minutos  lo  cual  no  se  pudo
cumplir  y  aunque  es  exagerada  esa  posible  reducción  de
tiempo,  si  podría  cumplirse  algo  más  plausible.  En  este
momento  aun  estando  en  una  calle  cuya  cartelería  permite
circular a 60 Km/h  [1], se tiene una velocidad promedio de 25
Km/h (según varios viajes realizados) lo cual tiene muchísimo
para mejorar.
La intendencia luego de diversas protestas de la gente que
vivía en los barrios por los que pasa el corredor de Garzón da
marcha atrás a proyectos similares como el corredor Agraciada
Norte[2],  y  General  Flores,  admitiendo  que  en  Garzón  no  se
había cumplido el objetivo[3].

La  obra  consistió  en  ensanchar  las  calles  de  Garzón,
manteniendo  las  dos  vías  para  cada  sentido  y  agregando  un
corredor central exclusivo para los ómnibus Urbanos (que no
salen de Montevideo), este es el llamado Corredor de Garzón.
Garzón  maneja  un  conjunto de 6 de  semáforos por  cruce,
teniendo  un  cruce  cada  300  metros  aproximadamente.  Se
pretende  mejorar  el  estado  de  la  sincronización  de  los
semáforos en  una región de Garzón. Junto a esto se intentará
minimizar los tiempos de espera que se puedan obtener en el
presente
\subsubsection{Objetivos}
Mostrar que los algoritmos evolutivos pueden solucionar problemas complejos en escenarios  reales, siendo una herramienta perfectamente utilizable.
\subsubsection{limitaciones}
La informacion exacta sobre la configuracion de los semaforos y la densidad de trafico no esta disponible publicamente, por tanto el relevamiento de datos hecho en situ tiene fue realizado para un numero determinado de calles y dias. 
\subsubsection{Enfoque}
Se creara un programa que implemente un algoritmo genetico  y llama a un simulador de trafico para obtener las metricas a optimizar.
\subsubsection{Contribuciones de este trabajo}
\subsubsection{Contacto con el publico}
Cabe destacar que este proyecto se presento en Ingeniera de muestra, siendo bien recibido por el publico. Donde constatamos de primera mano que la problematica es real y nosotros como Ingenieros tenemos las herramientas necesarias para solucionar problemas que afectan directamente a la sociedad.
\subsubsection{Estructura del documento}
El capitulo 2 brinda el marco teorico necesario para poder comprender los siguentes capitulos sobre la imprelmentacion del algoritmo y las simulaciones. Ademas que se da un pantallazo al problema del trafico.
El capitulo 3 mostramos los trabajo relacionados haciendo incapie en algoritmos geneticos para la sincronizacion de semaforos.
el capitulo 4 explicamos la estrategia seguida para la resolucion del problema y damos en detalle el disenio de la misma.
El capitulo 5 cuenta con tablas, graficas e informacion relacionada con el analisis experimental realizado en los distintos escenarios que elegimos.
El capitulo 6 da las conclusiones finales y el trabajo a futuro que se puede realizar.






