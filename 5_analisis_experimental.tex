\chapter{Análisis Experimental}
Esta sección presenta la evaluación experimental del Algoritmo Evolutivo multiobjetivo propuesto para resolver el problema de sincronización de semáforos del corredor Garzón. Inicialmente se presenta una descripción de los escenarios utilizados en la evaluación experimental y la plataforma de ejecución en la que se llevo a acabo la experimentación. A continuación se detalla el análisis experimental, que se divide en dos etapas: primero se analiza la configuración paramétrica para encontrar una mejor calidad de resultados y luego se realizan los experimentos de validación para los que se reportan los resultados numéricos y comparaciones.


\section{Plataforma de ejecución y Desarrollo}
Los algoritmos fueron desarrollados usando la biblioteca Malva que fue extendida en el código base para soportar la creación de nuevos hilos de ejecución para lograr el funcionamiento en paralelo.

El análisis experimental fue realizado en la infraestructura Cluster Fing (sitio web: http://www.fing.edu.uy/cluster), en un procesador AMD Opteron 6272 de 2.09GHz, 48Gb RAM, 64 cores, corriendo el sistema operativo CentOS Linux 5.2.


\section{Ajuste paramétrico}
Se busca la mejor configuración inicial de los parámetros realizando pruebas experimentales con diferentes combinaciones. Los elementos que se ajustaran serán los siguientes: el tiempo de simulación, el criterio de parada, el tamaño de la población y la probabilidad de mutación y cruzamiento.

Para estas pruebas se generan tres escenarios de tráfico diferentes. En este caso no se utilizan datos recabados de la realidad, como el tráfico vehicular o la frecuencia de los ómnibus, para no sesgar el algoritmo a un caso en particular
Los tres escenarios de tráfico son los siguientes:

\begin{itemize}
	\item Tráfico Bajo: 30 ómnibus y 500 vehículos	
	\item Tráfico Medio: 60 ómnibus y 1000 vehículos
	\item Tráfico Alto: 120 ómnibus y 2000 vehículos
\end{itemize}


En el ajuste paramétrico primero se define el tiempo de simulación y el criterio de parada, luego de establecidos se realizan las pruebas para todas las combinaciones de tasa de cruzamiento y mutación buscando los mejores valores para ajustar el algoritmo.

Al estar utilizando un cluster tenemos disponible tanto la métrica del tiempo real que llevo la ejecución, así como también el tiempo secuencial, es decir la suma del tiempo de procesamiento de todos los procesadores involucrados en la evaluación del algoritmo. 

Cuando se realizan comparaciones en los tiempos de ejecución se utiliza el tiempo secuencial que es independiente de la cantidad de procesadores utilizados en las pruebas. Las diferentes ejecuciones fueron realizadas sobre el Cluster utilizando entre 4 y 32 procesadores ya que por la naturaleza de recursos compartidos no siempre se tiene una cantidad igual  de procesadores libres y para estas pruebas el número de procesadores no es relevante. Si lo será cuando se realice el análisis de eficiencia computacional que se estudiará más adelante.



\subsection{Pesos de la función fitness}

Para las siguientes pruebas la función \emph{fitness} del algoritmo (\ref{eq:funcion_fitness}) tendrá los pesos \emph{x = y = 1}. Esto da pesos equitativos tanto a ómnibus como a otros vehículos por lo que no existe prioridad para uno u otro. Más adelante se realizarán experimentos con otras variantes.


\subsection{Tiempo de simulación}
El tiempo de simulación refiere a la duración de una simulación usando SUMO. Es un parámetro que se puede configurar y permite tener un mejor control sobre los tiempos totales de ejecución del algoritmo.

Se establece un tiempo de simulación de 4000 \emph{steps} (medida interna de tiempo del simulador). Este número representa 66 minutos en la simulación, mientras el tiempo real de ejecución depende de la plataforma y de la cantidad de vehículos, pero se encuentra entre los 5 a 20 segundos. Se tuvo en cuenta y validó que en cada escenario más del 80 \% de los vehículos completaran la simulación, es decir que llegaran a sus destinos. Se realizaron 10 ejecuciones del algoritmo para cada uno de los tres tipos de tráfico comprobando que se cumplía el criterio.


\subsection{Criterio de parada}
Se elige como criterio de parada el número de generación, lo cual permite estandarizar las pruebas para una mejor comparación. Para determinarlo se busca un compromiso entre un buen resultado y un tiempo de ejecución apropiado que no sea excesivo.

Para esto se decide que por un lado la ejecución del algoritmo deberá estar comprendida entre 1 y 24 horas, y además comprobar experimentalmente que el valor de \emph{fitness} no tiene una gran variación en las ultimas 100 generaciones. Se ejecutaron 10 ejecuciones del algoritmo por cada tipo de tráfico.

Luego de la realización de las pruebas se elige el número de 500 generaciones como criterio de parada.
En la siguiente gráfica se aprecia como el valor de \emph{fitness} no presenta grandes variaciones luego de la generación 400, además el tiempo de ejecución real está dentro del margen pautado. La gráfica no muestra todos los valores, solo algunos representativos para una mejor visualización.



\begin{figure}[h]
\centering
\includegraphics[width=0.8\linewidth]{Figures/criterio_parada}
\caption{Resumen representativo de ejecuciones del algoritmo para establecer el criterio de parada.}
\label{fig:criterio_parada}
\end{figure}



\subsection{Tamaño de la población}

Para la elección de la población se tendrán en cuenta tres elementos: el valor de \emph{fitness} encontrado, el tiempo de ejecución total y la plataforma de ejecución.

La máxima cantidad poblacional a estudiar está determinada por la infraestructura del Cluster Fing, donde la cantidad máxima de procesadores en el mismo nodo es 64. Teniendo en cuenta que la mejor distribución de trabajo es un individuo de la población por procesador, se determina que la máxima cantidad de población será de 64 individuos.

Luego se eligen dos valores mas para completar el análisis, estos son 32 y 48 individuos por población. Se tiene en cuenta que no son lo suficientemente bajos y son valores con los que se obtiene una distribución adecuada de individuos en la infraestructura. Para las pruebas se realizan 10 ejecuciones del algoritmo por cada tipo de tráfico obteniendo el promedio de esos valores.

La tabla \ref{table:parametro_poblacion} muestra los resultados obtenidos; como se aprecia no existen grandes diferencias en la elección de un número poblacional sobre otro. Por tanto se elige como número de población 32, teniendo en cuenta que el tiempo de ejecución secuencial del algoritmo es el menor y que insume menos recursos al ejecutarse. Esto es importante por utilizar el Cluster Fing que es utilizada por otras personas y con recursos limitados.

\begin{table}[h]
	\renewcommand{\arraystretch}{1.2}
	\caption{Comparación de fitness para distintas poblaciones}
	\label{table:parametro_poblacion}
	\centering
	\begin{tabular}{ccrrcp{2cm}}
		\hline
	    \multirow{2}{*}{\textbf{Población}}& & 
		\multicolumn{2}{c}{\textbf{Fitness}} \\
		\cline{3-4}
		& & {mejor} 
		& {promedio} 
		& \textbf{Tiempo ejecución serial (m)} \\
		\hline
		32 & & {17.28} & 16.37$\pm$0.5 & 10184$\pm$526\\
		48 & & {16.19} & 15.84$\pm$0.3 & 6772$\pm$256\\
		64 & & {17.27} & 16.46$\pm$0.6 & 4853$\pm$155\\
		\hline
	\end{tabular}
\end{table}





\subsection{Probabilidad de mutación y cruzamiento}

Para configurar la probabilidad de cruzamiento (pc) se consideraron tres valores candidatos (0.5, 0.8, y 1) y para la probabilidad de mutación (pm) otros tres (0.01,  0.05,  y  0.1). De las nueve combinaciones posibles, se realizaron tres ejecuciones independientes  del  algoritmo para cada uno de los tres tipos de tráfico (bajo, medio y alto).  
 
 \begin{table}[H]
 	\renewcommand{\arraystretch}{1.2}
 	\caption{Combinaciones de probabilidad de cruzamiento(pc) y de mutación (pm)}
 	\label{table:parametro_mutacion_cruzamiento}
 	\centering
 	\begin{tabular}{p{1cm}p{1cm}p{3.5cm} }
 		\hline
 		$p_C$& 
 		$p_M$ & 
 		Fitness promedio  $\pm$ desviación estándar\\ 
 		\hline
 		0.5 & 0.01  &  16.09$\pm$0.30\\
 		0.5 & 0.05 &  15.60$\pm$0.17\\
 		0.5 & 0.1  &  16.16$\pm$0.42\\
 		0.8 & 0.01  &  16.04$\pm$0.55\\
 		0.8 & 0.05  &  15.85$\pm$0.32\\
 		0.8 & 0.1  &  16.08$\pm$0.34\\
 		1 & 0.01 &  16.08$\pm$0.45\\
 		1 & 0.05 &  15.82$\pm$0.34\\
 		1 & 0.1 &  16.04$\pm$0.25\\
 		\hline
 	\end{tabular}
 \end{table}
 
Analizando la tabla y la gráfica se puede apreciar claramente que para una probabilidad de mutación de 0.05 se obtienen los peores resultados. Otro dato interesante es que no existe gran diferencia en el resto de las combinaciones.

\begin{figure}[H]
	\centering
	\includegraphics[width=0.8\linewidth]{Figures/grafica_mutacion_cruzamiento}
	\caption{Gráfica con combinaciones de probabilidad de cruzamiento(pc) y de mutación (pm)}
	\label{fig:grafica_mutacion_cruzamiento}
\end{figure}


Se comprueba que todas las muestras siguen la distribución normal para poder aplicar el test de Student. 

%Siendo las hipótesis de la prueba u1 el promedio del  grupo 1 y u2 el del grupo 2)
%
%
%        \begin{equation}
%        \label{eq:student_eq}
%			\begin{split}
%				H0: u1  = u2  \\
%				H1: u1 != u2 
%			\end{split}			
%        \end{equation}

Si hacemos una comparación entre la combinación del mejor promedio (0.5-0.1) y del peor (0.5-0.05) con el test de Student obtenemos t(x) = 0.07 que nos indica que para un nivel de significancia de  0,1 la hipótesis nula es rechazada, por lo tanto existe evidencia estadística para elegir la combinación con el mejor promedio (0.5-0.1) sobre la combinación con el peor promedio (0.5-0.05) para la ejecución del algoritmo.

Para comprobar si es la mejor opción  se toman las dos combinaciones con el mejor promedio (0.5-0.1) y (0.5-0.01) obteniendo en el test Student t(x) = 0.71 que nos indica que no existe una diferencia significativa entre ambas muestras por lo que elegir una sobre otra no implicaría grandes beneficios.

En tal sentido podríamos elegir cualquiera de las dos, en este caso se elige  la combinación (0.5-0.01) por su buen promedio y baja desviación estándar.



\section{Descripción de escenarios}
En esta sección se presentan los escenarios que serán evaluados, el primero es el escenario base que representa la realidad actual y el segundo un escenario alternativo que contiene modificaciones con el objetivo de mejorar la realidad.

\subsection{Caso base o realidad actual del corredor}
Esto representa la situación actual en términos de tráfico, red vial y sincronización de semáforos del corredor Garzón. El objetivo es realizar una simulación para recabar mas información (velocidad promedio de vehículos y ómnibus) que sirva para comparar con los resultados del algoritmo utilizando los datos obtenidos in-situ (configuración de semáforos, cantidad de vehículos, etc). 

Se valida su correctitud comparando los tiempos obtenidos en la simulación con tiempos obtenidos in-situ de los recorridos de ida y vuelta para los vehículos. Para el caso de los ómnibus se utilizan las frecuencias, cantidad y recorridos de los mismos que son de acceso público.

Se realizó un estudio sobre datos proporcionados por la IMM que contenían el posicionamiento de los ómnibus, velocidad instantánea y datos de la línea durante todo el día para una semana en particular. De esta forma se constató que para las líneas de ómnibus que pasan por Garzón la velocidad promedio de los ómnibus es de 14.5 km/h.

Esto permitió calibrar el escenario modificando aspectos de la simulación relacionados con los ómnibus para mayor precisión.

Sobre este escenario geográfico se realizarán tres escenarios de tráfico : bajo, medio y alto.
El caso medio representa los datos obtenidos, el bajo es disminuyendo el 50\% de vehículos y el tiempo de espera en las paradas de ómnibus teniendo en cuenta que en este caso existirá menos personas utilizando el transporte público. Las frecuencias de ómnibus se mantienen iguales ya que no son alteradas en la realidad.
El caso de tráfico alto se aumenta 50 \%  los vehículos y el tiempo de espera en la parada de los ómnibus.

El aumento y disminución del 50\% se obtuvo al analizar datos proporcionados por la IMM de la zona de Garzón de años anteriores. \newline
En resumen tenemos los siguientes escenarios de tráfico:

\begin{itemize}
\item Tráfico Alto:  3000 vehículos en la simulación y 70 ómnibus. 
\item Tráfico Medio: 2000 vehículos y 70 ómnibus.
\item Tráfico Bajo:  1000 vehículos y 70 ómnibus.
\end{itemize}

 

%\subsection{Escenario Evolutivo }
%En este caso se ejecuta el algoritmo evolutivo sobre el caso base para obtener una nueva sincronización de semáforos optimizada que repercutirá en la calidad del tráfico.

%En este caso se realizan 20 ejecuciones independientes sobre cada instancia de tráfico.


\subsection{Escenario alternativo}

Para mostrar la utilidad que tienen las simulaciones sobre un escenario real, se realiza un escenario alternativo. Una de las ventajas principales es que no requiere gran inversión monetaria, de tiempo y que no afecta la situación actual de la realidad, por lo que se pueden generar distintas pruebas para encontrar aquellas que logren un beneficio.

Analizando aquellos puntos que se entienden podrían atentar contra el buen funcionamiento del Corredor, se agregan algunas modificaciones al escenario base para intentar mejorarlo. 

El objetivo no es demostrar que será la mejor alternativa, sino dar una de las muchas alternativas que se pueden generar y probar con la simulación si se logran mejoras. Ya que pueden existir limitaciones o reglas que no estamos tomando en cuenta y que deben cumplirse en la realidad.

Entre los cambios estudiados se encuentran: eliminación de paradas y pasajes peatonales, alternar paradas y modificación de reglas de semáforos.



\section{Resultados}
Se muestran los resultados obtenidos tanto de la simulación de la realidad, como de la aplicación del algoritmo. Se presenta la simulación del escenario alternativo y la posterior evaluación. Además se realizan estudios sobre cambios en la función de fitness del algoritmo y un breve análisis de la eficiencia computacional.


\subsection{Valores numéricos del caso base}

En la tabla \ref{table:resultado_caso_base} se pueden ver las métricas obtenidas para las diferentes instancias de tráfico simulado para el caso base. Estos datos representan la realidad actual del Corredor Garzón. Como se aprecia la velocidad promedio de los ómnibus en tráfico medio es de 14.6km/h siendo 14.5km/h el valor que se obtuvo de analizar los datos reales proporcionados por la IMM, lo que verifica que el modelo se aproxima a la realidad. 
 
 \begin{table}[H]
 	\renewcommand{\arraystretch}{1.2}
 	\caption{Resultados del caso base mostrando la velocidad promedio ómnibus (vpb) y velocidad promedio vehículos(vpv) para los distintos tipos de tráfico}
 	\label{table:resultado_caso_base}
 	\centering
 	\begin{tabular}{p{2.5cm}p{2.5cm}p{2.5cm}p{2cm} }
 		\hline
 		&
 		$vbp(km/h)$& 
 		$vvp(km/h)$ & 
 		Fitness \\ 
 		\hline
 		Tráfico Bajo & 15.89  & 32.45& 13.42\\
 		Tráfico Medio & 14.59  & 28.81& 12.05\\
 		Tráfico Alto & 14.31  & 26.36& 11.30\\

 		\hline
 	\end{tabular}
 \end{table}
 
 En este caso no se representa la desviación standard ya que los resultados de la simulación del caso base son constantes, pues siempre se están probando los mismos recorridos de los vehículos y la configuración de los semáforos no cambia. Cuando se aplica el algoritmo obtenemos un valor diferente de velocidad y \emph{fitness} en cada ejecución ya que tienen distintas configuraciones de semáforos, en este caso se muestra la desviación standard.


\subsection{Resultados numéricos de la evaluación }

Como se aprecia en la tabla \ref{table:resultado_caso_base}, el algoritmo mejora la velocidad promedio tanto de ómnibus como de otros vehículos en los tres tipos de tráfico estudiados. Además la velocidad media de los vehículos se mantiene en un rango mucho más ajustado que en el caso original al variar el tráfico. Las mejoras logradas en el \emph{fitness} son de de hasta 24\%. A continuación se describe el análisis estadístico para comprobar la mejora.


\begin{table}[H]
	\renewcommand{\arraystretch}{1.2}	
		\centering
	\caption{Resultados luego de ejecutado el algoritmo mostrando velocidad promedio ómnibus (vpb) y  de otros vehículos(vpv) para los distintos tipos de tráfico }
	\label{table:resultado_caso_algoritmo}
	\begin{tabular}{cccccccc}
		\hline 
		Tráfico& 
		$vbp(km/h)$& 
		$vpv(km/h)$&
		\multicolumn{2}{c}{\emph{Fitness}}&  & 
		\multicolumn{2}{c}{Mejora \emph{fitness} (\%)}\\  \cline{4-5} \cline{7-8}&     &     & \multicolumn{1}{c}{Promedio} & \multicolumn{1}{c}{Mejor} &  & \multicolumn{1}{c}{Promedio} & \multicolumn{1}{c}{Mejor} \\ \hline
		Bajo & 17.92$\pm$0.18 & 34.30$\pm$0.40 & 14.50$\pm$0.14 & 14.88 & & 8.04 & 10.8  \\
		Medio& 16.95$\pm$0.32 & 33.29$\pm$0.29 & 13.95$\pm$0.15 & 14.19 & & 15.70& 17.7\\ 
		Alto & 16.51$\pm$0.61  & 32.90$\pm$0.25& 13.72$\pm$0.17 & 14.04 & & 21.40& 24.2\\	
		\hline	    
	\end{tabular}
\end{table}

Se realizaron 20 ejecuciones independientes para cada tipo de tráfico comprobando que siguieran una distribución normal.

Por tanto se puede aplicar el criterio de significancia estadística para validar los resultados. Este indica que el
algoritmo A es mejor que B si los resultados de A y B cumplen:

\begin{equation}
\label{eq:funcion_significancia}
\left |f_{avg}(A) - f_{avg}(B)  \right | > max(std(f_A),std(f_B))
\end{equation}

En este caso A representa el algoritmo y B el caso base. Esto indica que la diferencia del resultado promedio del algoritmo restado al resultado del caso base debe ser mayor a la máxima desviación.
Esto se cumple para todos los casos, por lo que se puede afirmar  que existe evidencia estadística para indicar que los resultados del algoritmo son mejores al del caso base.

\begin{figure}[H]
	\centering
	\includegraphics[width=0.8\linewidth]{Figures/duracion_viajes}
	\caption{Comparación de la duración en minutos de los viajes para ómnibus y otros vehículos en el recorrido completo del corredor Garzón para los diferentes tipos de tráfico.}
	\label{fig:duracion_viajes}
\end{figure}

Un resultado interesante es que la duración original de los viajes cuando el tráfico es bajo es casi igual a la duración de los viajes con tráfico alto luego de ejecutar el algoritmo.
Para ómnibus tenemos 24.5m y 23.6m y  para otros vehículos 12.0m y 11.9m.

\subsection{Detalles del escenario alternativo}
Los cambios propuestos incluyen eliminación de paradas, semáforos, pasajes peatonales y alternar paradas. Se estudiaron otras propuestas pero fueron descartadas por la poca viabilidad real de las mismas, como por ejemplo construir calles paralelas a Garzón o nuevas reglas en los cruces como existen en otros países.

\subsubsection{Eliminación de paradas}
Se consideraron dos paradas a eliminar que cumplieran con algunas características: no fueran cercana a una calle principal, que existiera otra parada cercana y que la eliminación de la parada no afecte en demasía a la gente en un traslado mayor.
En este caso se seleccionaron las paradas en la calle Ariel y Casavalle.

\subsubsection{Eliminación de pasajes peatonales}
Hay tres pasajes peatonales en el corredor con semáforos que detienen el tráfico, dos de ellos solo manejan una esquina (sin pulsador en funcionamiento) donde en el escenario alternativo se implementó solamente mediante un \emph{pare} en la calle transversal al corredor y el otro es netamente peatonal frente a la Facultad de Agronomía que fue totalmente eliminado. Una opción que mantiene los pasajes peatonales así como también los resultados obtenidos en el escenario alternativo sería implementar el pasaje peatonal por encima del corredor. Al eliminar los pasajes peatonales se aumenta la velocidad media de todo el transporte.

\subsubsection{Alternar paradas}

Uno de los problemas del ómnibus es su baja aceleración por lo que cada vez que este frena en un semáforo o en una parada demora en retomar una velocidad aceptable. Por tanto al reducir la cantidad de paradas que un ómnibus tiene que hacer se mejora la velocidad promedio.
La línea G recorre a Garzón de punta a punta, es cubierta por las empresas Coectc y Cutcsa. Una posibilidad de alternancia de paradas consiste en dividir las paradas por empresa y compartir las ganancias del corredor u otro método para equiparar el pasaje transportado. 

\begin{figure}[H]
	\centering
	\includegraphics[width=0.9\linewidth]{Figures/paradas_alternativas}
	\caption{Gráfico de paradas alternativas. Gris: Parada Eliminada. Azul: línea G de Coectc y de Cutcsa. Rojo: línea G de Coectc. Violeta: G de Cutcsa. - Imagen original extraída de montevideo.gub.uy}
	\label{fig:paradas_alternadas}
\end{figure}

Una empresa se detendrá en las paradas pares y la otra en las impares y algunas de mayor aglomeración de pasaje/tiempo serán realizadas por las dos. Cada empresa viajará por el corredor a 4 minutos de frecuencia (como en la actualidad). Si se reduce el número de paradas que hace un ómnibus, aumentará su velocidad promedio y no se deberá resentir en demasía el servicio ya que la disminución de la frecuencia en una parada se contrarresta con el aumento promedio de velocidad.

Este cambio es el que más aumenta la velocidad media y tal vez es uno de los más sencillos de implementar en la realidad.



\subsubsection{Cambio básico de semáforos}
Al hacer el relevamiento de los datos se encontró que en todas las intersecciones en donde una línea de ómnibus que circula por el corredor tiene un viraje a la izquierda, se hace detener el tránsito de la derecha de la misma, cada vez que el corredor central tiene la luz verde. Esto no parece tener mucho sentido ya que podrían seguir circulando por el corredor sin ningún tipo de problema, el carril que hay que detener es el de la izquierda del ómnibus cuando una línea dobla a la izquierda pero no los dos carriles al mismo tiempo.

No se tiene conocimiento si esto corresponde a un error en la configuración, un tema de costos o facilidad para manejar los dos semáforos de los carriles paralelos juntos. Como esto ocurre en varias intersecciones y en ambos sentidos este cambio mejora la velocidad promedio de los autos que circulan por los dos carriles.

Este cambio se aplicó en las siguientes intersecciones:
\begin{itemize}
	\item Islas Canarias: dobla línea 409 hacia la izquierda, orientado a Colón (Norte).
	\item Camino Ariel: doblan líneas como la  2 y la 148 hacia la izquierda, orientado a Paso Molino (Sur). 
	\item Camino Casavalle: dobla línea 174 hacia la izquierda, orientado a Paso Molino (Sur). 
\end{itemize}

\subsection{Valores numéricos al aplicar los cambios}

Para determinar cuales son los cambios que logran mejores rendimientos se elabora la tabla \ref{table:resultado_alternativo}. Esta se basa en el tráfico medio, ya que solo se quiere realizar una comparación sencilla de las mejoras realizadas. Estos cambios son acumulativos, por lo que se hacen uno después del otro. Se puede apreciar que el que logra una mayor diferencia es la utilización de paradas alternadas.


\begin{table}[H]
	\renewcommand{\arraystretch}{1.2}
	\caption{Valores del escenario alternativo con su velocidad promedio ómnibus (vpb) y velocidad promedio vehículos(vpv) comparando el \emph{fitness} para el tráfico medio }
	\label{table:resultado_alternativo}
	\centering
	\begin{tabular}{p{3.5cm}p{2.5cm}p{2.5cm}p{2cm}p{2cm} }
		\hline
		&
		$vbp(km/h)$& 
		$vvp(km/h)$ & 
		\emph{Fitness} &
		Mejora(\%)
		\\ 
		\hline
		Base & 14.59  & 28.81& 12.05 & -\\
		Eliminar Paradas & 15.44  & 29.03& 12.35 & 2.4\\
		Eliminar Peatonales  & 16.02  & 29.32& 12.59 & 4.4\\
		Paradas alternadas  & 19.17  & 28.88& 13.34 & 10.7\\	
		Cambio reglas  & 18.50  & 29.70& 13.39 & 11.1\\				
		\hline
	\end{tabular}
\end{table}

Una vez que se aplican todos las modificaciones sobre el escenario, se realiza un análisis para los demás tipos de tráfico. Como se ve en la tabla \ref{table:mejoras_trafico_alternativo} se obtienen mejores rendimientos en todos los tipos de tráfico estudiados y el mejor rendimiento se obtiene cuando el tráfico es alto. 

\begin{table}[H]
	\renewcommand{\arraystretch}{1.2}
	\caption{Mejoras obtenidas para las velocidades promedio de los ómnibus(vpb) y de otros vehiculos (vpv) en el escenario alternativo para distintos tipos de tráficos }
	\label{table:mejoras_trafico_alternativo}
	\centering
	\begin{tabular}{p{3.5cm}p{2.5cm}p{2.5cm}p{2cm}p{2cm} }
		\hline
		&
		$vbp(km/h)$& 
		$vvp(km/h)$ & 
		Fitness &
		Mejora \emph{fitness}(\%)
		\\ 
		\hline

		Tráfico Bajo & 20.72  & 33.18 & 14.97 & 11.5\\
		Tráfico Medio & 18.50  & 29.70& 13.39 & 11.1 \\
		Tráfico Alto  & 18.60  & 27.17& 12.7 & 12.6\\		
		\hline
	\end{tabular}
\end{table}

Una vez que tenemos este escenario alternativo se procede a aplicarle el algoritmo cuyo resultado veremos a continuación.


\subsection{Resultados de la evaluación sobre el escenario alternativo}

El escenario alternativo supuso una mejora sustancial en comparación con el caso base (11 \% en el valor de fitness). Se procede a aplicarle el algoritmo para determinar si aún hay posibilidad de mejorar los valores de velocidad y \emph{fitness}.


Los resultados obtenidos en la tabla \ref{table:mejoras_trafico_alternativo_algoritmo}  mejoran claramente el rendimiento del escenario alternativo y por supuesto del caso base en todos los tipos de tráfico. Comparando con la realidad actual se logran mejoras de hasta 37\%. 

Al comparar los resultados obtenidos se aprecia que cuanto más densidad de tráfico, mayor es el porcentaje de mejora. Además un resultado interesante es que las diferencias entre los valores de los distintos tipos de tráfico se redujo.




\begin{table}[h]
	\renewcommand{\arraystretch}{1.2}
	\caption{Mejoras obtenidas al aplicar el algoritmo sobre el escenario alternativo. Comparando las velocidades de ómnibus(vpb), otros vehículos(vpv) y el fitness con cada tipo de tráfico contra el caso base o realidad actual.}
	\label{table:mejoras_trafico_alternativo_algoritmo}
	\centering
	\begin{tabular}{cccccccc}
		\hline 
		Tráfico& 
		$vbp(km/h)$& 
		$vpv(km/h)$&
		\multicolumn{2}{c}{\emph{Fitness}}&  & 
		\multicolumn{2}{c}{Mejora \emph{fitness} (\%)}\\  \cline{4-5} \cline{7-8}&     &     & \multicolumn{1}{c}{Promedio} & \multicolumn{1}{c}{Mejor} &  & \multicolumn{1}{c}{Promedio} & \multicolumn{1}{c}{Mejor} \\ \hline

		Bajo & 23.15$\pm$0.36 & 34.43$\pm$0.33 & 15.99$\pm$0.08 & 16.10 & & 19.1& 19.90 \\
		Medio & 21.83$\pm$0.50  & 33.89$\pm$0.22 & 15.47$\pm$0.09& 15.65 & & 28.3 & 29.87\\
		Alto & 21.46$\pm$0.54  & 33.41$\pm$0.38 & 15.24$\pm$0.19& 15.50 & & 34.8 & 37.10\\	
		\hline		    
	\end{tabular}
\end{table}

En la gráfica \ref{fig:duracion_viajes_alernativo} se puede apreciar la comparación en la duración de los viajes. Se produce una gran reducción en la duración de los viajes de los ómnibus en los tres tipos de tráfico, mientras para el caso de los vehículos la mayor diferencia ocurre cuando el tráfico es alto.

\begin{figure}[H]
	\centering
	\includegraphics[width=0.8\linewidth]{Figures/duracio_viajes_alternativo}
	\caption{Comparación de la duración en minutos de los viajes para ómnibus y otros vehículos en el recorrido completo del corredor Garzón para los diferentes tipos de tráfico. Al aplicar el algoritmo sobre el escenario alternativo.}
	\label{fig:duracion_viajes_alernativo}
\end{figure}

Otra vez podemos aplicar el criterio de significancia estadística( \ref{eq:funcion_significancia}) para comprobar que la mejora es significativa tanto al comparar con los valores del caso base como con los del alternativo.

\subsection{Variación de la función de \emph{fitness}}

La función \emph{fitness} (\ref{eq:funcion_fitness}) utilizaba los pesos \emph{x = y = 1} lo que representa un balance equitativo  para ómnibus y vehículos.

%Por un cruce de Garzón pasan cada hora:
%70 ómnibus , si aproximamos con 23 personas= 1610 personas por hora
%800 autos, si aproximamos  2 personas por vehículo nos da 1600 personas por hora.
%Una cantidad similar  pasan por el cruce en ambos medios de transporte por lo que no existe una tendencia a favor de una sobre la otra, se podría aproximar que 50\% eligen el ómnibus y 50\% el auto.

Estos pesos pueden ser variados en función de lo que se necesite, por lo que se realizaron pruebas con dos tipos de pesos para comparar como varían las velocidades cuando se da más peso a un tipo de vehículo sobre el otro.


\subsubsection{Prioridad ómnibus}
En este caso se le dará más prioridad a los ómnibus. Esto se sostiene en el hecho que uno de los objetivos buscados por la IMM  es que se utilice más el transporte colectivo como parte de su Plan de Movilidad Urbana \citep{PlanMovilidad}. Con la premisa que al mejorar la duración del viaje en ómnibus en relación al del auto por el corredor, las personas que utilizan auto para sus viajes optarán por el transporte colectivo.

Por tanto se experimentó cambiando los pesos de la función \emph{fitness} con un peso de 70\% para los ómnibus y 30\% al resto de los vehículos.


\subsubsection{Prioridad a otros vehículos}

En este caso, se asignó 70\% del peso a los vehículos y 30\% a los ómnibus. Es el caso opuesto al anterior y resulta útil para poder comparar como varían los valores de las velocidades.

\subsubsection{Resultados}

La siguiente tabla compara las velocidades promedio de ómnibus y vehículos para los tres tipos de pesos que se probaron y por cada tipo de tráfico.  El caso 50-50 es el caso base donde los pesos son iguales, 70-30 es el caso con más prioridad para los ómnibus y el 30-70 más prioridad a los otros vehículos. Se analiza cuanto varían las velocidades de ómnibus (var. vpb) y otros vehículos(var. vpv) comparando contra el caso 50-50 de cada tipo de tráfico.


\begin{table}[H]
	\renewcommand{\arraystretch}{1.2}
	\caption{Modificación de los pesos para ómnibus (pb) y para otros vehículos (pv) en la función \emph{fitness}. Analizando las variaciones en la velocidad promedio de ómnibus (vpb),  otros vehículos (vpv) y \emph{fitness}. }
	\label{table:analisis_fitness}
	\centering
	\begin{tabular}{p{1cm}p{1.2cm}p{1.8cm}p{1.8cm}p{1.8cm}p{1.2cm}p{1.2cm}p{1.2cm} }
		\hline
		Tráfico &
		pb(\%) pv (\%)& 
		vpb & 
		vpv &
		fitness &
		var. \newline vpb(\%) &
		var. \newline vpv(\%) &
		var. \newline \emph{fitnesss}(\%)
		\\ 
		\hline
		& 50-50  & 17.92$\pm$0.18 & 34.30$\pm$0.40 & 14.50$\pm$0.14  &- & - & -\\		
		Bajo & 70-30  & 17.93$\pm$0.23 & 34.06$\pm$0.17 & 12.65$\pm$0.11  & +0.07 & -0.70 & -12.79\\		
		& 30-70 & 17.55$\pm$0.23 & 34.71$\pm$0.21 & 16.42$\pm$0.10  & -2.06 & +1.18 & +13.21\\
		\hline
		
		& 50-50  & 16.95$\pm$0.32 & 33.29$\pm$0.29 & 13.95$\pm$0.15  &- & - & -\\		
		Medio & 70-30  & 17.29$\pm$0.27 & 33.08$\pm$0.14 & 12.24$\pm$0.12  & +2.0 & -0.62 & -12.30\\		
		& 30-70 & 16.71$\pm$0.42 & 33.79$\pm$0.31 & 15.92$\pm$0.11  & -1.41 & +1.49& +14.11\\
		
		\hline
		& 50-50  & 16.51$\pm$0.60 & 32.90$\pm$0.25 & 13.72$\pm$0.17  &- & - & -\\		
		Alto & 70-30  & 16.72$\pm$0.14 & 32.79$\pm$0.26 & 13.75$\pm$0.07  & +1.24 & -0.33 & +0.19\\	
		& 30-70 & 15.48$\pm$0.42 & 33.20$\pm$0.25 & 15.49$\pm$0.16  & -6.23 & +0.92 & +12.87\\
		\hline
	\end{tabular}
\end{table}


Los resultados indican que al variar los pesos de la función \emph{fitness} las velocidades promedio de los vehículos se ve afectada. En el caso de dar más prioridad a los ómnibus se produce como cabía esperar un aumento en su velocidad promedio y una leve baja en la velocidad promedio del resto de los vehículos. Cuando el tráfico es bajo este cambio casi no aumenta la velocidad de los ómnibus. Una explicación posible de este comportamiento es que ya se llegó a un limite máximo y no se puede mejorar más.

Al dar más prioridad a los otros vehículos se produce un aumento en su velocidad y una disminución en la velocidad de los ómnibus la cual es muy evidente en el caso de tráfico alto. Este resultado permite apreciar como estos valores son fuertemente afectados por la densidad de tráfico que se estudie.

En general las variaciones en las velocidades no son grandes pero suficientemente apreciable para tener cierta libertad al plantear distintos objetivos que tiendan a favorecer un tipo u otro de vehículos.


\subsection{Eficiencia computacional}

Se realiza un estudio de la eficiencia computacional del algoritmo para analizar los tiempos de ejecución cuando se usan varios procesadores y como se comporta su capacidad de paralelismo.

Se evalúan nueve ejecuciones del algoritmo; tres con con cada tipo de tráfico: alto, medio y bajo, para estudiarlo en diferentes contextos. El algoritmo utiliza 32 hilos de ejecución por lo que utilizamos esa cantidad de procesadores.

Las pruebas fueron realizadas sobre el node40 del Cluster Fing, con un procesador AMD Opteron 6272 2.09GHz, 48 GB RAM y 32 \emph{cores} utilizados.

El \emph{speedup} (S) mide la mejora de rendimiento de una aplicación al aumentar la cantidad de procesadores comparando con el rendimiento al usar un solo procesador.
\begin{equation}
\label{eq:funcion_speedup}
S = \frac{T_1}{T_N}
\end{equation}
Donde ${T_1}$ es el tiempo de ejecución del algoritmo serial o secuencial, y ${T_N}$ el tiempo del algoritmo ejecutado sobre N procesadores.
\newline

La eficiencia computacional (E) corresponde al valor normalizado del \emph{speedup} (entre 0 y 1) respecto a la cantidad de procesadores. Los valores cercanos a uno indican una alta eficiencia computacional.
\begin{equation}
\label{eq:funcion_eficiencia}
E = \frac{T_1}{N*T_N} = \frac{S}{N}
\end{equation}



\begin{table}[H]
	\renewcommand{\arraystretch}{1.2}
	\caption{Análisis de eficiencia computacional comparando los tiempos de ejecución en serial y paralelo en minutos. }
	\label{table:analisis_speedup}
	\centering
	\begin{tabular}{p{2.5cm}p{2.5cm}p{2.5cm}p{2.5cm}p{2.5cm} }
		\hline
		
		Instancia& 
		Serial(m) & 
		Paralelo(m) &
		\emph{Speedup} &
		Eficiencia
		\\ 
		\hline
		bajo1  & 1572 & 59 & 26.64 & 0.83\\
		bajo2  & 1571 & 59 & 26.62 & 0.83\\
		bajo3  & 1183 & 44 & 26.88 & 0.84\\
		
		medio1  & 3002 & 119 & 25.22 & 0.78\\
		medio2  & 2195 & 82 & 26.76 & 0.83\\
		medio3  & 3007 & 120 & 25.05 & 0.78\\
		
		alto1  & 2920 & 110 & 26.5 & 0.82\\
		alto2  & 4365 & 183 & 23.85 & 0.74\\
		alto3  & 4276 & 177 & 24.15 & 0.75\\
		\hline
		  &  & Promedio & 25.7$\pm$1.1 & 0.80$\pm$0.03\\
		
		\hline
	\end{tabular}
\end{table}


El algoritmo paralelo logra una mejora sustancial en los tiempos de ejecución con un valor promedio del \emph{speedup} de 25.7  y  eficiencia promedio de 0.8, lo cual puede considerarse como buenas métricas.

\begin{figure}[H]
	\centering
	\includegraphics[width=0.8\linewidth]{Figures/speedup1}
	\caption{Comparación de los \emph{speedup} promedios para cada tipo de tráfico. Se representa el caso serial que corresponde al \emph{speedup} = 1 para fines de comparación.}
	\label{fig:speedup1}
\end{figure}

En la gráfica se observa como a medida que aumenta el tráfico disminuye el speedup. Esto sucede por que está influenciado por los accesos al disco duro, al tener más vehículos circulando en la simulación se tiene que leer y escribir más información en los archivos, lo que aumenta el tiempo de ejecución del algoritmo, aunque como se ve no tiene un gran impacto.