\chapter{Análisis Experimental}
En esta sección describimos los distintos escenarios que vamos a probar y los resultados obtenidos en cada uno de ellos así como comparativas que consideramos relevantes.

\section{Descripción de escenarios}

\subsection{Caso base}
Esto representa la situación actual en términos de trafico, red vial y sincronización de semáforos del corredor Garzón. 

\subsection{Escenario Inicial }
En este caso ejecutamos nuestro algoritmo evolutivo sobre el caso base para obtener una nueva sincronización de semáforos optimizada que repercutirá en la calidad del trafico.

\subsection{Escenario Alternativo}
Luego de analizar aquellos puntos que creemos atentan contra el buen funcionamiento del Corredor, agregamos algunas modificaciones al escenario base para intentar mejorarlo. 
Estos son limitar el numero de cruces en los que se puede doblar a la izquierda, 



\section{Desarrollo y Plataforma de ejecución }
Los algoritmos fueron desarrollados usando la librería Malva que fue extendida en el código base para soportar la creación de nuevos hilos de ejecución para lograr el funcionamiento en paralelo.



Los escenarios paralelos fueron ejecutados en el cluster fing:

Cluster: Es un conjunto de computadoras independientes conectadas para que trabajen integradas como un solo sistema. De esta forma se consigue un alto rendimiento en la ejecución de tareas. 

Cluster Fing: Es una infraestructura de alto desempeño, que brinda soporte en la resolución de problemas complejos que demandan un gran poder de computo.

Descripcion del hardware: 
\begin{itemize}
	\item 9 servidores de cómputo
	\subitem Quad core Xeon E5430, 2x6 MB caché, 2.66GHz, 1.333 MHz FSB.
	\subitem 8 GB de memoria por nodo.
	\subitem Adaptador de red dual (2 puertos Gigabit Ethernet).
	\subitem  Arquitectura de 64 bits.
	\subitem Servidor de archivos: 2 discos de 1 TB, capacidad ampliable a 10 TB.
	\subitem Nodos de cómputo: discos de 80 GB.
	\item Switch de comunicaciones
	\subitem Dell Power Connect, 24 puertos Gigabit Ethernet.
	\item Switch KVM (16 puertos) y consola.
	\item UPS APC Smart RT 8000VA.
\end{itemize}

\section{Ajuste de parámetros de algoritmos}
Buscamos la mejor configuración inicial de los parámetros realizando pruebas experimentales con diferentes combinaciones.  Estos son: el tamaño de la población,  probabilidad de mutación, probabilidad de cruzamiento, etc.
Para esto se realizaron 20 ejecuciones independientes para el algoritmo secuencial inicial.
El criterio de parada se eligió por el numero de generación.

Para la población se probaron 32, 64, 128 . Los resultado indicaron que 

Para el cruzamiento 0.5, 0.8, 1
Para mutación 0.01, 0.05 y 0.1

La mejor  configuración obtenida fue:
Población:120, mutación:0.01 , cruzamiento: 1

Las gráficas muestras el promedio de las 20 ejecuciones para resolver el escenario inicial.


\subsection{Tiempo de simulación}

Cada simulación tiene una duración fijada en base a ajustes previos que realizamos para que al menos 80 % de los vehiculos emitidos al inicio llegaron a su destino con un 90% de confianza (este parametro lo explicamos en ajuste de parametros? hay q poner algo de shapirowilk de confianza).
Esto nos permite mantener constante el tiempo de ejecución total del algoritmo y saber cuanto demorara su ejecución teniendo en cuenta la población y generaciones configuradas. De esta forma se logra una mayor confianza a la hora de comparar los resultados.



\section{Resultados}
Presentaremos los resultados obtenidos  utilizando los parámetros óptimos obtenidos para el escenario inicial, el escenario modificado, y la prueba en el cluster.

\subsection{Resultado simulación caso base}
\subsection{Resultado Escenario Inicial ( Algoritmo  Secuencial)}
\subsection{Resultado Escenario Alternativo ( Algoritmo  Secuencial)}
\subsection{Resultado de ejecución en paralelo. }

\subsection{Comparación caso base vs Algoritmo Secuencial}
Análisis comparativo : test paramétrico
H1)  los  resultados  de  mejor  fitness  tienen  una distribución normal.
H2)Existe  una  diferencia  significativa  entre  los  de conjuntos  de  muestras  obtenidos  por  el  algoritmo  y  la realidad.

El  test  de  normalidad  de  Shapiro-Wilks  resultó  ser verdadero  en  ambos  casos  con  un  alto  porcentaje  de
confiabilidad y luego al realizar los test T-student [] resulto tener menos de 0,0001 lo que se considera una diferencia que
estadísticamente  es  significativa.  Esto confirma algo que resulta evidente ya que al comparar
el promedio de los mejores fitness con la realidad encontramos
una diferencia de un 31,8% y un 15% respectivamente y estos
casos  son  bastante  acotados  ya  que  hay  mucha  cantidad  de
vehículos con grandes/medias distancias.

Con  los  resultados  obtenidos  en  las  ejecuciones  para  los
escenarios  1  y  2  se  obtuvo  un  mejor  fitness  de  661  y  515
respectivamente, comparado con el tiempo de la configuración
real que demora 1019 y 690, las soluciones son muy buenas.



\subsection{Comparación Algoritmo Secuencial vs Algoritmo Paralelo}
Speedup se define como ...

\subsection{Escalabilidad de Algoritmo Paralelo }
Ejecutar en 4, 8 ,16, 32

\section{Resumen}

