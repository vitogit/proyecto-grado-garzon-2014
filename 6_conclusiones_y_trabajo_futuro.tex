\chapter{Conclusiones y trabajo futuro}

\section{Conclusiones}
A pesar de que el problema de sincronización de semáforos
es un problema muy difícil de abordar, los resultados obtenidos
muestran la capacidad de los algoritmos genéticos para abordar
problemas de este tipo, obteniendo resultados muy buenos. Si
bien el algoritmo planteado tiene ciertas limitaciones como por
ejemplo es dependiente del tráfico que exista, se pueden hacer
estudios sobre el tráfico para tener resultados más adaptados a
la realidad. También se podría cambiar el enfoque y plantear el
problema  como  un  problema  multiobjetivo,  cada  individuo
(que  representa  la  configuración  de  semáforos)  se  podría
evaluar  con  distintos  flujos  de  tránsito,  pudiendo  así  definir
distintos criterios sobre cómo evaluar los individuos.
\section{Trabajo futuro}
