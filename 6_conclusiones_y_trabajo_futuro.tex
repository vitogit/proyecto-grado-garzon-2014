\chapter{Conclusiones y trabajo futuro}

\section{Conclusiones}
Los objetivos planteados al principio del trabajo fueron cumplidos.
Se encontró información útil al realizar la investigación sobre los trabajos relacionados que ayudó a mejorar la solución presentada. 
Al estudiar el problema del tráfico se constató que realmente afecta a la población y al desarrollo de las ciudades, por tanto es imprescindible la búsqueda de nuevas soluciones. A nivel de nuestro país no se encontraron soluciones similares, por lo que este trabajo es un aporte interesante que demuestra que existen tanto las herramientas como el conocimiento necesario para realizarlo.
 
Las simulaciones demostraron su valor al dar la flexibilidad de probar distintas variantes de forma sencilla y poder crear un escenario alternativo con modificaciones agregadas que logra una mejora del 11 \% en el valor de fitness.
 
A pesar de que el problema de sincronización de semáforos es un problema difícil de abordar, los resultados obtenidos muestran la capacidad de los algoritmos genéticos para resolver problemas de este tipo, obteniendo evidencia estadística de que logra mejorar la situación actual. En general el algoritmo logra una mejora de hasta  24.2 \% (21.40 \% en promedio) de el valor de fitness comparando con la realidad actual, mientras el escenario alternativo obtiene una mejora de hasta 37.1 \% (34.7\% en promedio) en el valor de fitness.

El enfoque multiobjetivo aún siendo básico dio la flexibilidad para analizar las diferentes velocidades medias de ómnibus y otros vehículos, lo que permitió realizar comparaciones independientes que dotaron al trabajo de un mayor nivel de detalle.

El desarrollo de algoritmos con capacidad de paralelización son fundamentales sobre  todo en problemas complejos que requieren mucho poder de computo como el que se abordo. El algoritmo obtuvo buenas métricas de speedup sin las cuales hubiera sido muy difícil generar la cantidad de pruebas presentadas.

\section{Trabajo futuro}

La elaboración de los mapas para la simulación requiere modificaciones para que sean reconocidos por el simulador, en algunos casos se tuvo que realizar manualmente ya que las herramientas no brindaban la granularidad necesaria.
Además el agregado de la configuración de semáforos, de las líneas y paradas de ómnibus puede ser un proceso lento y propenso a errores, por tanto para un futuro se sugiere la creación o búsqueda de herramientas que automaticen o agilicen este trabajo.

El algoritmo puede ser aplicado a otros lugares con solo cambiar los datos de entrada: mapa, tráfico, configuración de los semáforos y recorrido de ómnibus. Dado el alcance del trabajo solo se enfocó en la zona del Corredor Garzón pero sería interesante aplicarlo en otros escenarios para determinar su rendimiento.

Los trabajos de  \citet{Montana1996} y \citet{Vogel2000}  proponen la adaptabilidad del algoritmo en tiempo real, aunque esto requiere del agregado de sensores a la red puede ser un método de mejora importante sobre todo en zonas de gran densidad de tráfico.
