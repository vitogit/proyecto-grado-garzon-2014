\chapter{Conclusiones y trabajo futuro}

\section{Conclusiones}
A pesar de que el problema de sincronización de semáforos es un problema difícil de abordar, los resultados obtenidos
muestran la capacidad de los algoritmos genéticos para resolver problemas de este tipo, obteniendo resultados muy buenos. 

El enfoque multiobjetivo aún siendo básico dio buenos resultados en el sentido de priorizar un tipo de trafico u otro.

El desarrollo de algoritmos con capacidad de paralelización son fundamentales sobre  todo en problemas complejos que requieren mucho poder de computo como el que se abordo. Y demuestran que son útiles en acelerar el procesamiento.
\section{Trabajo futuro}

La elaboración de los mapas para la simulación donde se incluye cambios en las rutas para que sean reconocidas por el simulador, agregado de la configuración de semáforos, agregado de las lineas y paradas de ómnibus puede ser un proceso lento y tedioso, por tanto para un futuro se podría sugerir la realización de herramientas que automaticen o agilicen este trabajo.

Los trabajos de  \citep{Montana1996} y \citep{Vogel2000}  proponen la adaptabilidad del algoritmo en tiempo real, aunque esto requiere del agregado de sensores a la red puede ser un método de mejora importante sobre todo en zonas de gran densidad de trafico.
