\chapter{Conclusiones y trabajo futuro}

\section{Conclusiones}
Este trabajo presenta una solución al problema de sincronización de semáforos en el Corredor Garzón utilizando un AE multiobjetivo.

El problema del tráfico afecta a la población y al desarrollo de las ciudades. En la actualidad, se considera imprescindible la búsqueda de soluciones innovadoras y eficientes para mejorar la calidad de servicio ofrecida por las infraestructuras viales urbanas. En nuestro país (Uruguay), no existen antecedentes de propuestas relacionadas con la optimizacion del trafico urbano. En este contexto, este trabajo plantea un aporte interesante, orientado a estudiar el problema y demostrar que existen las herramientas y el conocimiento necesario para resolverlo.

La sincronización de semáforos es un problema difícil de abordar, en especial al trabajar sobre escenarios reales que revisten complejidad. Sin embargo, las técnicas de inteligencia computacional permiten calcular soluciones eficientes al problema. 
El AE propuesto en este trabajo contempla un amplio número de características del problema que proporcionan realismo y aplicación práctica a la solución aportada, incluyendo informacion real de mapas y datos de tráfico obtenidos en relevamientos in-situ y mediante análisis de trazas de GPS.

El enfoque multiobjetivo permitió resolver el problema mediante un enfoque de optimización global, estudiando las velocidades medias de ómnibus y otros vehículos en la zona del Corredor Garzón. El análisis experimental se enfocó en realizar un análisis comparativo con el caso base que representa la realidad actual del Corredor Garzón. Los resultados obtenidos muestran la capacidad de los AE para resolver el problema: el AE propuesto logra una mejora de hasta \textbf{15.3}\% en la velocidad promedio de ómnibus y \textbf{24.8}\% en la velocidad promedio de otros vehículos, obteniendo evidencia estadística que valida las mejoras calculadas. 

Las simulaciones del tráfico demostraron su utilidad al dar la flexibilidad necesaria para probar distintas variantes de forma sencilla, y poder diseñar un escenario alternativo con modificaciones agregadas al escenario base con el objetivo de mejorar las velocidades promedio de los vehiculos. La aplicacion del AE propuesto sobre este escenario alternativo logra una mejora de hasta \textbf{49.9}\% en la velocidad promedio de ómnibus y \textbf{26,7}\% en la velocidad promedio de otros vehículos comparando con la realidad actual.

El desarrollo de AE con capacidad de paralelización son fundamentales, sobre todo en problemas complejos que requieren mucho poder de computo como el abordado. El AE obtuvo buenas métricas de \emph{speedup} sin las cuales hubiera sido muy difícil realizar la cantidad de pruebas presentadas.

\section{Trabajo futuro}

El diseño de mapas para la simulación del tráfico requiere la realización de modificaciones para que sean reconocidos por el simulador, en algunos casos se aplicaron manualmente ya que las herramientas no brindaban la granularidad necesaria. Además, la edición de los archivos que representan la configuración de semáforos, las líneas y paradas de ómnibus supone un proceso lento y propenso a errores. Por estos motivos, para un futuro, se sugiere el desarrollo o búsqueda de nuevas herramientas que automaticen o agilicen este trabajo.

El AE desarrollado, puede ser aplicado a otros lugares geográficos, modificando  los datos de entrada: mapa, tráfico, configuración de los semáforos y recorrido de ómnibus. El alcance del trabajo sólo se enfocó en la zona del Corredor Garzón pero sería interesante aplicarlo en otros escenarios para determinar su rendimiento.

Los trabajos de  \citet{Montana1996} y \citet{Vogel2000} proponen la adaptabilidad del algoritmo en tiempo real, aunque esto requiere del agregado de sensores a la red, podría resultar en una mejora importante sobre todo, en zonas de gran densidad de tráfico.
