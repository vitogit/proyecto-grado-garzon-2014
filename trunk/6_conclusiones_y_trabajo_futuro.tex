\chapter{Conclusiones y trabajo futuro}

\section{Conclusiones}
Al analizar los objetivos planteados al inicio del proyecto, se puede afirmar que las principales metas fueron cumplidas exitosamente.

La investigación de los trabajos relacionados permitió estudiar en profundidad las técnicas de computación evolutiva utilizadas para resolver el problema de sincronización de semáforos, enfocándose en algoritmos evolutivos. En esta etapa, se relevó la información necesaria para formar una base solida en la que se sustenta el diseño de la solución presentada. Se realizó un estudio general sobre la problemática del tráfico, que afecta tanto a la población, como al desarrollo de las ciudades. En la actualidad, se considera imprescindible la búsqueda de soluciones innovadoras y eficientes para mejorar la calidad de servicio ofrecida por las infraestructuras viales urbanas. En nuestro país (Uruguay), no existen antecedentes de propuestas relacionadas con la optimización del trafico urbano. En este contexto, este trabajo plantea un aporte interesante, orientado a estudiar el problema y demostrar que existen las herramientas y el conocimiento necesario para resolverlo.

La sincronización de semáforos es un problema difícil de abordar, en especial al trabajar sobre escenarios reales que revisten complejidad. Sin embargo, las técnicas de inteligencia computacional permiten calcular soluciones eficientes al problema. Se realizaron instancias realistas del problema que modelan la situación actual del Corredor Garzón. Estas instancias contemplan un amplio número de características del problema que proporcionan realismo y aplicación práctica a la solución aportada. El diseño del mapa se basó en información real de la zona, obteniendo el mapa base del servicio OSM. Los datos de tráfico fueron relevados in-situ, siguiendo los lineamientos recomendados en la bibliografía estudiada y de las recomendaciones obtenidas de las reuniones con funcionarios de la división de tránsito de la IMM. Se tuvieron en cuenta las frecuencias reales de las lineas de ómnibus que recorren el Corredor Garzón,  la ubicación correcta de las paradas de ómnibus y las velocidades promedio de los vehículos que fue calculada a partir del análisis de trazas de GPS proporcionadas por la IMM.
 
El enfoque multiobjetivo permitió resolver el problema mediante un enfoque de optimización global, estudiando las velocidades medias de ómnibus y otros vehículos en la zona del Corredor Garzón. El análisis experimental se enfocó en realizar un análisis comparativo con el caso base que representa la realidad actual del Corredor Garzón. Los resultados obtenidos muestran la capacidad de los AE para resolver el problema: el AE propuesto logra una mejora de hasta \textbf{15.3}\% en la velocidad promedio de ómnibus y \textbf{24.8}\% en la velocidad promedio de otros vehículos, obteniendo evidencia estadística que valida las mejoras calculadas.

Se realizaron pruebas experimentales basadas en variar los pesos de la función de \emph{fitness} con el objetivo de dar mas prioridad a un tipo de vehículo sobre otro. Por ejemplo, en el caso de Garzón es interesante estudiar el caso donde se dá más prioridad a los ómnibus sobre el resto de los vehículos, ya que uno de los objetivos del Plan de Movilidad es promover el uso del transporte publico y se considera que al aumentar su velocidad relativa, aumentaría también su uso por parte de la población. Los resultados de estos experimentos mostraron como las velocidades promedio se ven afectadas, aunque las variaciones no son grandes, por ejemplo, al dar mas prioridad a los ómnibus en una instancia de tráfico medio, se logró una mejora en su velocidad promedio del 2\% y una disminución del resto de los vehículos de 0.62\%, comparando con el caso donde las prioridades de ómnibus y otros vehículos son iguales.

Las simulaciones del tráfico demostraron su utilidad al dar la flexibilidad necesaria para probar distintas variantes sobre los escenarios de forma sencilla, y poder diseñar un escenario alternativo realizando modificaciones al escenario base con el objetivo de mejorar las velocidades promedio de los vehículos. Estas modificaciones incluyen: la eliminación de paradas y pasajes peatonales, la modificación de reglas básicas de los semáforos y la alternancia de paradas. La aplicación del AE propuesto sobre el escenario alternativo logra una mejora de hasta \textbf{49.9}\% en la velocidad promedio de ómnibus y \textbf{26,7}\% en la velocidad promedio de otros vehículos comparando con la realidad actual.

El desarrollo de AE con capacidad de paralelización es fundamental, sobre todo en problemas complejos que requieren mucho poder de cómputo, como el abordado en este proyecto. El AE obtuvo buenas métricas de \emph{speedup} lo que permitió reducir considerablemente el tiempo de ejecución del AE al comparar con el caso secuencial. Sin ésta reducción en el tiempo de ejecución del AE hubiera sido muy difícil realizar la cantidad de pruebas presentadas.

\section{Trabajo futuro}
Durante el desarrollo del proyecto, se detectaron varias líneas de trabajo e investigación que seria interesante abordar como trabajo futuro.

El diseño de mapas para la simulación del tráfico requiere la realización de modificaciones para que sean reconocidos por el simulador, en algunos casos se aplicaron manualmente ya que las herramientas no brindaban la granularidad necesaria. Además, la edición de los archivos que representan la configuración de semáforos, las líneas y paradas de ómnibus supone un proceso lento y propenso a errores. Por estos motivos, para un futuro, se sugiere el desarrollo o búsqueda de nuevas herramientas que automaticen o agilicen este trabajo.

El AE desarrollado puede ser aplicado a otros lugares geográficos, modificando  los datos de entrada: mapa, tráfico, configuración de los semáforos y recorrido de ómnibus. El alcance del trabajo sólo se enfocó en la zona del Corredor Garzón, pero sería interesante aplicar el AE en otros escenarios con diferentes características (en el tráfico, en las vías de tránsito, en la cantidad de semáforos, etc) para determinar su capacidad para mejorar las velocidades de los vehículos bajo esas circunstancias.

Un aspecto interesante para desarrollar en el futuro, es el uso de un enfoque multiobjetivo explícito para compararlo con el de agregación lineal utilizado en este proyecto. Para implementar este algoritmo evolutivo multiobjetivo, se podría considerar el uso de otro \emph{framework}, por ejemplo jMetal que incluye varios tipos de algoritmos multiobjetivos.

Los trabajos de  \citet{Montana1996} y \citet{Vogel2000} proponen la adaptabilidad del algoritmo en tiempo real, aunque esto requiere del agregado de sensores a la red, podría resultar en una mejora importante sobre todo, en zonas de gran densidad de tráfico. Se instalarían sensores en los semáforos que detecten la cantidad de automóviles en las intersecciones en tiempo real y un sistema que procesara los datos obtenidos en varias intersecciones para optimizar la circulación del tráfico, modificando la configuración de las luces de los semáforos. El reto de este método se encuentra en que al ser en tiempo real, la configuración de semáforos tiene que ser obtenida relativamente rápida, por lo que el algoritmo utilizado tendría que ser adecuado a esta necesidad.
