\chapter{Conclusiones y trabajo futuro}

\section{Conclusiones}
Luego de analizar los objetivos planteados al inicio del trabajo se puede afirmar que se cumplieron satisfactoriamente.
La investigación de los trabajos relacionados permitió encontrar información relevante que ayudó a mejorar la solución presentada. 
Al estudiar el problema del tráfico, se constató  que afecta a la población y al desarrollo de las ciudades. Por este motivo se considera imprescindible la búsqueda de nuevas soluciones. En el contexto de nuestro país (Uruguay), no se encontraron soluciones similares, por lo que este trabajo se plantea como un aporte interesante, que demuestra que existen las herramientas y el conocimiento necesario para realizarlo.
 
Las simulaciones del tráfico demostraron su utilidad al dar la flexibilidad necesaria para probar distintas variantes de forma sencilla, y poder diseñar un escenario alternativo con modificaciones agregadas que logró una mejora del 11 \% en el valor de \emph{fitness} comparando con el escenario base que representa la realidad.
 
A pesar de que el problema de sincronización de semáforos es un problema difícil de abordar, los resultados obtenidos muestran la capacidad de los algoritmos genéticos para resolver problemas de este tipo, obteniendo evidencia estadística que valida las mejoras producidas. En general el algoritmo genético propuesto logra una mejora de hasta  24.2 \% (21.40 \% en promedio) del valor de fitness comparando con la realidad actual, mientras el escenario alternativo obtiene una mejora de hasta 37.1 \% (34.7\% en promedio) en el valor de fitness.

El enfoque multiobjetivo, aún siendo básico, permitió analizar las diferentes velocidades medias de ómnibus y otros vehículos, que fueron utilizadas para realizar análisis comparativos que enriquecieron el trabajo.

El desarrollo de algoritmos evolutivos con capacidad de paralelización son fundamentales, sobre todo en problemas complejos que requieren mucho poder de computo como el abordado. El algoritmo evolutivo obtuvo buenas métricas de \emph{speedup} sin las cuales hubiera sido muy difícil realizar la cantidad de pruebas presentadas.

\section{Trabajo futuro}

El diseño de mapas para la simulación del tráfico requiere la realización de modificaciones para que sean reconocidos por el simulador, en algunos casos se aplicaron manualmente ya que las herramientas no brindaban la granularidad necesaria. Además, la edición de los archivos que representan la configuración de semáforos, las líneas y paradas de ómnibus supone un proceso lento y propenso a errores. Por estos motivos, para un futuro, se sugiere el desarrollo o búsqueda de nuevas herramientas que automaticen o agilicen este trabajo.

El algoritmo evolutivo desarrollado, puede ser aplicado a otros lugares geográficos, modificando  los datos de entrada: mapa, tráfico, configuración de los semáforos y recorrido de ómnibus. El alcance del trabajo solo se enfocó en la zona del Corredor Garzón pero sería interesante aplicarlo en otros escenarios para determinar su rendimiento.

Los trabajos de  \citet{Montana1996} y \citet{Vogel2000} proponen la adaptabilidad del algoritmo en tiempo real, aunque esto requiere del agregado de sensores a la red, podría resultar en una mejora importante sobre todo, en zonas de gran densidad de tráfico.
