\chapter{Análisis Experimental}
En esta seccion describimos los distintos escenarios que vamos a probar y los resultados obtenidos en cada uno de ellos asi como comparativas que consideramos relevantes.

\section{Descripción de escenarios}

\subsection{Caso base}
Esto representa la situacion actual en terminos de trafico, red vial y sincronizacion de semaforos del corredor Garzon. 

\subsection{Escenario Inicial }
En este caso ejecutamos nuestro algoritmo evolutivo sobre el caso base para obtener una nueva sincronizacion de semaforos optimizada que repercutira en la calidad del trafico.

\subsection{Escenario Alternativo}
Luego de analizar aquellos puntos que creemos atentan contra el buen funcionamiento del Corredor, agregamos algunas modificaciones al escenario base para intentar mejorarlo. 
Estos son limitar el numero de cruces en los que se puede doblar a la izquierda, 



\section{Desarrollo y plataforma de ejecucion }
Los algoritmos fueron desarrollados usando la librería Malva que fue extendida en el codigo base para soportar la creacion de nuevos hilos de ejecucion para lograr el funcionamiento en paralelo

\section{Ajuste de parametros de algoritmos}
Buscamos la mejor configuracion inicial de los parametros realizando pruebas experimentales con diferentes combinaciones.  Estos son: el tamanio de la poblacion,  probabilidad de mutacion, probabilidad de cruzamiento, etc.
Para esto se realizaron 20 ejecuciones independientes para el algoritmo secuencial inicial.
El criterio de parada se eligio por el numero de generacion…

Para la poblacion se probaron 20, 50, 100 . Los resultado indicaron que 
Para el cruzamiento 0.5, 0.8, 1
Para mutacion 0.01, 0.05 y 0.1

La mejor  configuracion obtenida fue:
Poblacion:120, mutacion:0.01 , cruzamiento: 1

Las graficas muestras el promedio de las 20 ejecuciones para resolver el escenario inicial.


\section{Resultados}
Presentaremos los resultados obtenidos  utilizando los parametros optimos obtenidos para el escenario inicial, el escenario modificado, y la prueba en el cluster.

\subsection{Resultado simulacion caso base}
\subsection{Resultado Escenario Inicial ( Algoritmo  Secuencial)}
\subsection{Resultado Escenario Alernativo ( Algoritmo  Secuencial)}
\subsection{Resultado de ejecucion en paralelo. }

\subsection{Comparacion caso base vs Algoritmo Secuencial}
Analisis comparativo : test parametrico
H1)  los  resultados  de  mejor  fitness  tienen  una
distribución normal.
H2)Existe  una  diferencia  significativa  entre  los  de
conjuntos  de  muestras  obtenidos  por  el  algoritmo  y  la
realidad

El  test  de  normalidad  de  Shapiro-Wilks  resultó  ser
verdadero  en  ambos  casos  con  un  alto  porcentaje  de
confiabilidad y luego al realizar los test T-student [6] resulto
tener menos de 0,0001 lo que se considera una diferencia que
estadísticamente  es  extremadamente  significativa.  Esto
confirma algo que resulta un tanto evidente ya que al comparar
el promedio de los mejores fitness con la realidad encontramos
una diferencia de un 31,8% y un 15% respectivamente y estos
casos  son  bastante  acotados  ya  que  hay  mucha  cantidad  de
vehículos con grandes/medias distancias.
Con  los  resultados  obtenidos  en  las  ejecuciones  para  los
escenarios  1  y  2  se  obtuvo  un  mejor  fitness  de  661  y  515
respectivamente, comparado con el tiempo de la configuración
real que demora 1019 y 690, las soluciones son muy buenas.



\subsection{Comparacion Algoritmo Secuencial vs Algoritmo Paralelo (Calculo de SpeedUp)}

\subsection{Escabilidad de Algoritmo Paralelo }
Ejecutar en 4, 8 ,16, 32

\section{Resumen}

