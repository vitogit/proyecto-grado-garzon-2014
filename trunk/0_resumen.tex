{
\thispagestyle{empty}
~\\[0.2cm]
\begin{center}
    \textsc{\huge Algoritmos evolutivos en  } \\[0.2cm] 
    \textsc{\huge sincronización de semáforos en el  } \\[0.2cm]         
    \textsc{\huge Corredor de Garzón} \\[1cm]
    \textsc{\Large Resumen}
\end{center}
~\\[0.2cm]
\textbf{\large 
El proyecto propone el estudio de la sincronización de semáforos como problema de optimización multiobjetivo, y el diseño e implementación de un algoritmo evolutivo para resolverlo con alta eficacia numérica y desempeño computacional. \newline \newline
Se toma como aplicación la sincronización de semáforos en el Corredor de Garzón  (Montevideo, Uruguay). La cantidad de cruces, calles, tráfico y semáforos lo hace un problema interesante desde el punto de vista de su complejidad. Además se ha admitido, por parte de las autoridades responsables, los problemas relacionados con la sincronización de los semáforos, por lo que todavía hay espacio para la mejora de los tiempos promedio de los viajes.  \newline \newline
El problema de sincronización de semáforos es NP-difícil y no existe (hasta el momento) un método determinístico que lo resuelva. Se buscará mediante un algoritmo evolutivo llegar a una configuración aceptable de los semáforos maximizando la velocidad media tanto de ómnibus como de otros vehículos.
El enfoque seguido es la obtención de datos reales relacionados a la red vial, tráfico y configuración de semáforos y utilizar el simulador de tráfico SUMO para generar los datos requeridos por el algoritmo.
\newline \newline
El análisis experimental consiste en comparar los resultados del algoritmo con los valores obtenidos en la simulación de la realidad actual y además crear un escenario alternativo con modificaciones de la situación actual con el objetivo de obtener mejores métricas. Los resultados muestran que el algoritmo logra una mejora de hasta  24.2 \% (21.40 \% en promedio) el valor de fitness comparando con la realidad actual, mientras el escenario alternativo obtiene una mejora de hasta 37.1 \% (34.7\% en promedio) en el valor de fitness.
 } 	
	~\\[1.0cm]
    \textbf{\large Palabras clave: Corredor Garzón, simulación de tráfico,  Algoritmo Evolutivo, Sincronizacion semáforos, escenario real, Cluster}

}
\cleardoublepage
