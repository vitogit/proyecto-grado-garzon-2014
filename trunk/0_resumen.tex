{
\thispagestyle{empty}
~\\[0.1cm]
\begin{center}
    \textsc{\huge Algoritmos evolutivos aplicados a la  } \\[0.2cm] 
    \textsc{\huge sincronización de semáforos en el  } \\[0.2cm]         
    \textsc{\huge Corredor Garzón} \\[0.5cm]
    \textsc{\Large Resumen}
\end{center}
~\\[0.1cm]
\textbf{\large 
Este proyecto propone el estudio de la sincronización de semáforos como problema de optimización multiobjetivo, y el diseño e implementación de un algoritmo evolutivo para resolverlo con alta eficacia numérica y desempeño computacional. Se plantea como caso de estudio la sincronización de semáforos en el Corredor Garzón (Montevideo, Uruguay), un escenario urbano muy interesante por su complejidad, relacionada con el número de cruces, calles y semáforos, y por la problemática del tráfico en la zona. Además, las autoridades responsables admitieron la existencia de problemas relacionados con la sincronización de los semáforos, por lo que todavía hay espacio para la mejora de los tiempos promedio de los viajes en el Corredor Garzón.  \newline \newline
El problema de sincronización de semáforos es un problema de optimización NP-difícil por lo que los métodos computacionales exactos sólo son útiles en instancias de tamaño reducido. Este trabajo propone utilizar un algoritmo evolutivo para calcular una configuración eficiente de los semáforos, maximizando la velocidad media del transporte colectivo y de otros vehículos. El enfoque seguido comprende la obtención de datos reales relacionados con la red vial, el tráfico y la configuración de los semáforos, y la utilización del simulador de tráfico SUMO para generar los datos requeridos por el algoritmo evolutivo.
\newline \newline
El análisis experimental compara los resultados numéricos del algoritmo evolutivo con el escenario base que modela la realidad actual. Complementariamente se desarrolla un escenario alternativo cuyo objetivo es mejorar la velocidad promedio de ómnibus y otros vehículos realizando modificaciones sobre el escenario base. Los resultados demuestran que el algoritmo evolutivo propuesto logra mejoras significativas en la calidad de servicio al comparar con la realidad actual, mejorando hasta 15.3\% la velocidad promedio de ómnibus y 24.8\% la velocidad promedio de otros vehículos en el escenario base, mientras que al aplicar el algoritmo evolutivo en el escenario alternativo se obtienen mejoras de hasta 49.9\% en la velocidad promedio de los ómnibus y de hasta 26.74\% en la velocidad promedio de otros vehículos.
 } 	
	~\\[0.5cm]
    \textbf{\large Palabras clave: Algoritmo Evolutivo, Sincronización de semáforos,\newline Optimización de tráfico, Corredor Garzón}

}
\cleardoublepage
