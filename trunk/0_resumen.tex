{
\thispagestyle{empty}
~\\[0.2cm]
\begin{center}
    \textsc{\huge Algoritmos evolutivos en  } \\[0.2cm] 
    \textsc{\huge sincronización de semaforos en el  } \\[0.2cm]         
    \textsc{\huge Corredor de Garzón} \\[1cm]
    \textsc{\Large Resumen}
\end{center}
~\\[0.2cm]
\textbf{\large 
El proyecto propone el estudio de la Sincronización de Semáforos como problema de Optimización Multiobjetivo, y el diseño e implementación de un Algoritmo Genetico para resolver el problema con alta eficacia numérica y desempeño computacional en un escenario real utilizando simulaciones del trafico. \newline \newline
Se toma como aplicación la sincronización de semáforos del “Corredor de Garzón” (Montevideo, Uruguay) dado que luego de una gran inversión monetaria no se lograron los resultados esperados en cuanto a la optimización del tiempo o velocidad media del transporte. Ademas la cantidad de cruces, calles, trafico y cantidad de luces lo hace un problema interesante desde el punto de vista de su complejidad \newline \newline
El problema de sincronización de semáforos es NP- difícil y no existe (hasta el momento) un método determinístico que lo resuelva, se buscará mediante un algoritmo evolutivo llegar a una configuración aceptable de los semáforos para un conjunto de escenarios, minimizando los tiempos de espera de los vehículos y mejorando de esta manera la calidad del trafico.
Buscamos demostrar que estas tecnicas son aplicables a escenarios reales y que deberian considerarse en el abanico de posibilidades disponibles. } 	
	~\\[1.0cm]
    \textbf{\large Palabras clave: Algoritmo Evolutivo, Sincronizacion semaforos, escenario real, Cluster}

}
\cleardoublepage
