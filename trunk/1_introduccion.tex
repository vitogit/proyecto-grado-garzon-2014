\chapter{Introducción}
\epigraph{ \textit{He llamado a este principio, por el cual cada pequeña variación, si útil, es preservada, con el término de Selección Natural}}{--- Charles Darwin, El origen de las especies}

En esta sección se pretende introducir al lector en el contexto general donde se desarrolla este trabajo así como los objetivos buscados.

\section{Motivación y contexto}

Los algoritmos evolutivos han demostrado su utilidad en problemas complejos y particularmente uno de ellos es la sincronización de semáforos. Se busca desarrollar un algoritmo que logre resolver este problema con buenas métricas.
Por la flexibilidad inherente de este algoritmo es que no está destinado a resolver el problema en una zona en particular sino que se podría aplicar en forma general.

En este sentido se eligió la zona del corredor Garzón que presenta particularidades que la destacan y la hacen interesante desde el punto de vista de la investigación. Su complejidad viene dado por el largo del tramo, la cantidad de cruces, la complejidad y cantidad de semáforos en cada uno de ellos, las distintas reglas de tráfico aplicadas a cada tramo como por ejemplo exclusividad del ómnibus, o distinción para doblar a la izquierda, tráfico vehicular y transporte publico, calles no paralelas, entre otros.

Por tanto al probar que el algoritmo funciona en esta zona tan compleja se puede tener confianza de que se comportaría adecuadamente en otras zonas que no presentan tanta complejidad y obtener buenos resultados.



\newpage

\section{Objetivos}

Estos son los objetivos básicos que se plantearon al inicio del  proyecto.

\begin{itemize}
	\item Estudio del problema del tráfico y la sincronización de semáforos.
	\item Relevamiento de información sobre trabajos relacionados en este ámbito.
	\item Creación de un algoritmo evolutivo que resuelva el problema en la zona del corredor Garzón.
	\item Confección de el mapa y la configuración relativa a semáforos, tráfico y reglas de transito.	
	\item Demostrar que los algoritmos evolutivos pueden solucionar problemas complejos en escenarios  reales, siendo una herramienta perfectamente utilizable.
	\item Aplicar técnicas de computación de alto desempeño para aumentar el rendimiento de la solución.
	\item Probar la escalabilidad de la solución.
\end{itemize}

 
\section{Enfoque}

Desde un primero momento se intento dotar al proyecto de una buena aproximación de la realidad, en tal sentido se realizaron reuniones con el Ing. Juan Pablo Berta del Servicio de Ingeniería de Tránsito de la Intendencia de Montevideo en Agosto del  2014 y con el Ing. Daniel Muniz del departamento de Informática de la Intendencia en Setiembre de 2014 para conocer la situación del trafico capitalino, aprender de su experiencia y obtener datos que nos fueran útiles para el proyecto.


Buscando una aproximación aun mas precisa se realizaron trabajos de campo para determinar la configuración de los semáforos, la densidad de tráfico y el tiempo de recorrida. El mapa y la frecuencia de ómnibus son de acceso publico así como el simulador utilizado.

Se creará un programa que implemente un algoritmo evolutivo multiobjetivo que utiliza un simulador de tráfico para obtener las métricas a optimizar. Se busca obtener una nueva configuración de semáforos que en las simulaciones se comporte mejor que la situación actual basándonos en la velocidad promedio de ómnibus y el resto de los vehículos.

Dada la complejidad del problema el programa sera paralelo y se utilizara la plataforma Cluster fing para poder aumentar el tiempo real de procesamiento. además se intentara luego de estudiado los resultados buscar escenarios alternativos para mejorar los tiempos de viaje en el Corredor.

\section{Limitaciones y alcance}

La zona modelada comprende todo el tramo de el Corredor Garzón y dos caminos paralelos a ambos lados que dada la configuración de las calles las cuales corren en diagonal fue un proceso complejo.
El revelamiento de datos hecho in-situ fue realizado para un número determinado de calles ( se evalúa Garzón y 5 cruces representativos) para tener una aproximación útil no pretende ser un estudio detallado sobre el trafico en la zona.

Como lo que se pretende modelar es el tráfico vehicular y transporte publico por tanto la simulación de peatones no están incluidos.


\section{Aportes}

\begin{itemize}
	\item Se desarrollo el sitio web (??) donde los interesados podrán acceder para encontrar información sobre el proyecto y los resultados.
	\item Se realizó un \emph{paper} en idioma ingles de 10 paginas con el objetivo de presentarlo en conferencias internacionales.
	\item El proyecto fue presentado en Ingeniera demuestra 2014, siendo bien recibido por el publico. Constatando de primera mano que la problemática es real y llegando a la conclusión que los Ingenieros tienen las herramientas necesarias para solucionar problemas que afectan directamente a la sociedad.

\end{itemize}






\section{Estructura del documento}
En el capítulo 2 se hace un repaso sobre que es un algoritmo genético y los conceptos relacionados.
El capitulo 3 brinda el marco teórico donde se da una introducción al problema del tráfico y el corredor Garzón particularmente, además se presenta el simulador de trafico y otras herramientas utilizadas.
En el capitulo 4 se muestran los trabajo relacionados enfocando en algoritmos genéticos para la sincronización de semáforos.
En el capítulo 5 se explica la estrategia seguida para la resolución del problema y se da en detalle el diseño de la misma.
El capítulo 6 cuenta con tablas, gráficas e información relacionada con el análisis experimental realizado en los distintos escenarios que se eligieron.
El capítulo 7 da las conclusiones finales y el trabajo a futuro que se puede realizar.






