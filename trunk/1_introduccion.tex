\chapter{Introducción}
\epigraph{ \textit{He llamado a este principio, por el cual cada pequeña variación, si útil, es preservada, con el término de Selección Natural}}{--- Charles Darwin, El origen de las especies}

Este capítulo pretende introducir al lector en el contexto general donde se enmarca el problema de sincronización de semáforos en el Corredor Garzón. Inicialmente se describen las motivaciones y el enfoque seguido para su desarrollo . A continuación se especifican los objetivos propuestos, las limitaciones y su alcance. Para finalizar se delinea la estructura del documento con una breve descripción de cada capítulo. 

\section{Motivación y contexto}

En gran parte del mundo, el parque automotor esta creciendo de forma sostenida desde hace varios años. Lo que provoca serios problemas relacionados con la congestión del tráfico que afecta el desarrollo de las ciudades y la calidad de vida de las personas. Esto produce una progresiva disminución en la velocidad media de circulación y aumento del consumo de combustible que impacta directamente en la contaminación atmosférica y sonora.

Uruguay no escapa a este fenómeno, en particular su capital Montevideo. Aunque la situación no es tan critica como en otras ciudades del mundo las autoridades han tomado medidas para solucionar este problema, implementando el Plan de Movilidad que pretende mejorar la eficiencia del transporte público. Uno de sus puntos principales es la construcción de Corredores urbanos de tránsito con exclusividad para ómnibus.

Como parte del Plan de Movilidad Urbana se encuentra el Corredor Garzón, ubicado en la ciudad de Montevideo, con una extensión de 6.5km, 24 cruces semaforizados y vías exclusivas para ómnibus. Desde su inauguración en el año 2012 ha recibido criticas por no cumplir uno de sus principales objetivos de agilizar el transporte publico. La intendenta Ana Olivera admitió que la duración de los viajes aumentó considerablemente. En palabras al Diario \cite{olivera2015} indicó: 


\begin{quote}\small
	``En el momento más crítico llegaron a haber 24 minutos más, lo medimos, porque hubo un momento que se confundió mucho o se partidizó la crítica y nosotros queríamos tener los datos objetivos. Y aquí el dato objetivo era que los ciudadanos de la zona y en particular los de Lezica, no era que se sentían perjudicados, era que habían sido perjudicados. Y nosotros lo asumimos"
\end{quote}

Los métodos para optimizar el tráfico se pueden dividir en dos categorías, por un lado la modificación de las rutas, por ejemplo agregando nuevas vías de transito, o ensanchando las existentes. Esto provoca grandes mejoras pero como punto negativo exigen un alto costo monetario y espacio físico disponible. Por otro lado se encuentran los métodos para influir en el comportamiento de los conductores, en donde se encuentra la configuración de los semáforos o agregado de señalizaciones, entre otros. Estos métodos son muchas veces la única opción viable o disponible. Por tanto, estudiar soluciones para la sincronización eficiente de semáforos con el objetivo de mejorar la velocidad promedio de los viajes, se presenta como un aporte interesante y necesario para el desarrollo de las ciudades. 

El problema de sincronización de semáforos se considera NP-difícil, por lo que los métodos exactos son solo útiles en instancias de tamaño reducido. Por tanto es recomendable la utilización de métodos heurísticos que resuelvan el problema. Los algoritmos evolutivos han demostrado su utilidad en la resolución de problemas complejos por lo que lo convierten en un candidato interesante para resolver el problema de sincronización de semáforos. 

Se pretende crear un algoritmo evolutivo capaz de mejorar la velocidad promedio de los vehículos en la zona del Corredor Garzón cambiando la configuración de sus semáforos. Con el objetivo de aportar una solución eficiente e innovadora para el desarrollo de la ciudad y mejorar la calidad de vida de sus ciudadanos. Además las particularidades del Corredor Garzón lo convierten en un reto complejo desde el punto de vista de la investigación, esto viene dado por el largo del tramo, la cantidad de cruces, semáforos, reglas de exclusividad, diferente tipos de trafico, entre otras.



\section{Objetivos}

Estos son los objetivos que se plantearon al inicio del proyecto:

\begin{itemize}
	\item Estudio del problema del tráfico y la sincronización de semáforos.
	\item Relevamiento de información sobre trabajos relacionados en este ámbito.
	\item Creación de un algoritmo evolutivo multiobjetivo y paralelo que resuelva el problema en la zona del corredor Garzón.
	\item Creación de un mapa y configuración relativa a semáforos, tráfico y reglas de tránsito que sea precisa y obtenida de la realidad actual.	
	\item Aplicación de técnicas de computación de alto desempeño para aumentar el rendimiento de la solución.

\end{itemize}

 
\section{Enfoque}

Desde un principio se intentó dotar al proyecto de una buena aproximación de la realidad, en tal sentido se realizaron reuniones con el Ing. Juan Pablo Berta del Servicio de Ingeniería de Tránsito de la Intendencia de Montevideo en Agosto del  2014 y con el Ing. Daniel Muniz del departamento de Informática de la Intendencia en Setiembre de 2014 para conocer la situación del tráfico capitalino, aprender de su experiencia y obtener datos que fueran útiles para el proyecto. En este sentido se accedió a información del posicionamiento GPS y velocidad de los ómnibus para una semana en particular, lo que permitió procesarla para obtener la velocidad media de los ómnibus en la zona de Garzón. Además se obtuvo información sobre el conteo vehicular en varios puntos de la ciudad, que aunque no estaban actualizados fueron útiles para aprender como realizarlo.

Buscando una aproximación aún más precisa se realizaron trabajos de campo para determinar la configuración de los semáforos, la densidad de tráfico y el tiempo del recorrido. El mapa y la frecuencia de ómnibus son de acceso público así como el simulador utilizado.

Se creará un programa que implemente un algoritmo evolutivo multiobjetivo que utiliza un simulador de tráfico para obtener las métricas a optimizar. Se busca obtener una nueva configuración de semáforos que en las simulaciones se comporte mejor que la situación actual basándonos en la velocidad promedio de ómnibus y del resto de los vehículos.

Dada la complejidad del problema el algoritmo se ejecutará en paralelo y se utilizará la plataforma Cluster Fing para poder acelerar el tiempo real de procesamiento. Además se creará un escenario alternativo con modificaciones de la realidad actual con el objetivo de mejorar las métricas.

\section{Limitaciones y alcance}
Para resolver el problema se necesita crear un modelo aproximado de la realidad, de forma que pueda ser manejado por el algoritmo y las simulaciones. Al realizar este proceso hay elementos que no serán tenidos en cuenta y otros que se simplifican. 

El mapa desarrollado comprende todo el tramo del Corredor Garzón, dos caminos paralelos a ambos lados y todas las calles que lo cruzan. Fue importado del servicio \citep{OSM}, comprobando que tenia algunas inconsistencias con la realidad, por lo que se realizaron modificaciones tanto para ajustarlo como para que fuera compatible con el simulador, intentando mantener un compromiso entre la realidad y el modelo. 

Al no existir datos públicos sobre la configuración de los semáforos, se realizó un relevamiento manual de los mismos en el lugar, por tanto no se puede asegurar su exactitud, aunque se realizaron verificaciones para palear esta limitación. Algo similar sucede con la densidad de tráfico, no se encontraron datos actualizados, por lo que se realizaron conteos vehiculares en cinco cruces representativos buscando una aproximación útil y no un estudio detallado del tráfico en la zona. Por esta razón se simplifica contando solo los autos, sin tener en cuenta otros vehículos como camiones, motos o bicicletas. La cantidad de ómnibus y sus frecuencias fue tomado de información disponible públicamente.


\section{Aportes}
Además del presente proyecto se realizaron otras aportaciones que lo complementan. Se desarrolló un sitio web en la siguiente dirección: \url{http://www.fing.edu.uy/inco/grupos/cecal/hpc/AECG} donde los interesados podrán acceder a información sobre el proyecto y sus resultados. Además se realizó un \emph{paper} en idioma inglés de 10 páginas con el objetivo de presentarlo en conferencias internacionales. Se destaca que el proyecto fue presentado en Ingeniería deMuestra 2014, siendo bien recibido por el público. Constatando de primera mano que la problemática es real, llegando a la conclusión que los Ingenieros tienen las herramientas necesarias para solucionar problemas que afectan directamente a la sociedad.


\section{Estructura del documento}
En el siguiente capítulo se hace un repaso sobre fundamentos teóricos necesarios para comprender el resto del trabajo. Comienza describiendo el problema del tráfico y algunas formas de solucionarlo ente las que se encuentra la creación de planes de movilidad y la sincronización de semáforos. Se da información sobre corredores urbanos de tránsito y en concreto el Corredor Garzón, así como un repaso breve sobre algoritmos evolutivos y simuladores de tráfico. Además se muestran los trabajos relacionados haciendo especial foco en algoritmos genéticos para la sincronización de semáforos.

En el capítulo 3 se explica como se modela el problema y la arquitectura diseñada  para resolverlo. Luego se comenta el trabajo de campo realizado para obtener datos de la realidad utilizados en la solución. Además se detalla el algoritmo genético utilizado así como la biblioteca y herramientas usadas.

El capítulo 4 cuenta con la descripción de los escenarios, los resultados de la evaluación del algoritmo y las comparaciones realizadas. Se presenta el escenario que modela la situación actual y el escenario alternativo con modificaciones, los cuales se evaluarán con el algoritmo. Además se realizan pruebas para comprender como varían los resultados al modificar la función de fitness y se realiza un breve análisis de la eficiencia computacional del algoritmo.

El capítulo 5 da las conclusiones finales y el trabajo a futuro que se puede realizar.






