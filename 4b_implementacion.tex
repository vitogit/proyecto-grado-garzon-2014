\chapter{Implementación de la solución}

Este capítulo detalla la implementación del AE desarrollado. Comienza describiendo a la biblioteca Malva, utilizada para la implementación y continúa especificando el AE desarrollado donde se explica: la representación del cromosoma, la generación de la población, la función de \emph{fitness} y el funcionamiento de los operadores evolutivos. Para finalizar se detalla el modelo de paralelismo utilizado.

\section{Implementación: Biblioteca Malva}

Dada la complejidad del problema, se contempló desde el inicio del proyecto la utilización de un \emph{framework} para el desarrollo del AE. De esta forma, se facilita el desarrollo y la implementación de una solución robusta y flexible. Como se detalló en el marco teórico, existen varias opciones disponibles de \emph{framework} para el desarrollo de AEs. Por las particularidades del problema planteado, existen ciertos requerimientos a la hora de seleccionar un \emph{framework}, entre los que se destacan:

\begin{itemize}
	\item Código abierto y gratuito: Al ser un proyecto de investigación de índole académico es conveniente que no existan costos económicos asociados. Además, es importante que se cuente con el código abierto con el fin de introducir modificaciones en el código base o corregir errores.
	\item Algoritmo genético: El \emph{framework} debe facilitar el desarrollo de un algoritmo genético ya que es la base en la que se sustenta el proyecto.
	\item Algoritmo paralelo: La complejidad del problema y de las instancias que se estudian, justifican la necesidad de que el \emph{framework} incluya la opción de ejecución en paralelo para reducir los tiempos de ejecución. Si no cuenta con esta funcionalidad nativa, es deseable que al menos exista la posibilidad de modificar el código base para agregarla.
	\item Plataforma: Es requerido que el \emph{framework} se pueda ejecutar en la plataforma de computación de alto desempeño Cluster FING cuyo sistema operativo es Linux CentOs, ya que es donde se realiza el análisis experimental del trabajo. 
	\item Confiabilidad: Es deseable que el \emph{framework} sea lo suficientemente estable como para tener confianza de que el algoritmo funcionará de manera correcta. Se valorará positivamente que el \emph{framework} cuente con documentación accesible, ejemplos de código desarrollado, y casos de éxito que lo utilicen.
	\item Lenguaje: Es deseable que el \emph{framework} esté desarrollado en C++ o Java, por la experiencia que el equipo de desarrollo cuenta en estos lenguajes. 
	\item Multiobjetivo: Es deseable que el \emph{framework} soporte algoritmos multiobjetivo pero no requerido, ya que la función que se quiere implementar es sencilla y puede usarse una agregación lineal de los objetivos.
\end{itemize} 

Luego de analizar los puntos anteriores, se seleccionó la biblioteca Malva para la implementación del AE. Malva cumple con la mayoría de los puntos: es de código abierto y gratuito, facilita el desarrollo de un AG, funciona en la plataforma Cluster FING, utiliza el lenguaje C++ y el punto de mayor peso en la decisión fue la confiabilidad que daba saber que ya se habían realizado implementaciones exitosas de soluciones similares utilizando AE paralelos en la plataforma Cluster FING. Los dos puntos donde Malva no satisface totalmente los requerimientos son: el funcionamiento en paralelo y el soporte de algoritmos multiobjetivo. Estos problemas se pueden resolver de forma aceptable y sencilla como se explica a continuación. Para el caso del funcionamiento en paralelo al ser Malva de código abierto, es posible modificar el código para soportar la creación de hilos de ejecución, este procedimiento se explica en más detalle en la última sección de este capítulo que trata sobre el modelo de paralelismo y su implementación. El segundo punto acerca del soporte de algoritmos multiobjetivo por parte del \emph{framework}, no supone un gran inconveniente, ya que la función de \emph{fitness} que se plantea utilizar es una combinación lineal sencilla de objetivos y puede ser utilizada en Malva. 
A continuación se detalla su funcionamiento y características.

Como se explicó anteriormente, Malva \citep {Malva} surge como una variante del proyecto Mallba \citep{Mallba}. Malva propone la actualización, mejora y desarrollo de Mallba como un proyecto de código abierto colaborativo.  Su objetivo es proveer varios esqueletos de heurísticas de optimización que puedan ser utilizados y extendidos de manera fácil y eficiente. Los esqueletos se basan en separar dos conceptos: el problema concreto que se quiere resolver y el método utilizado para resolverlo. Por lo tanto, un esqueleto se puede ver como una instancia de una plantilla genérica para resolver un problema particular, manteniendo todas las funcionalidades genéricas.

Malva utiliza el lenguaje C++, con el objetivo de brindar modularidad y flexibilidad. Los esqueletos se ofrecen como un conjunto de clases \emph{requeridas} que el usuario deberá modificar para adaptarlo a su problema, y clases \emph{provistas} que incluyen todos los aspectos internos del esqueleto siendo  independientes del problema particular. Entre los algoritmos provistos se encuentran el AG, el algoritmo CHC \citep{CHC} y otros.


\section{Especificación del Algoritmo Genético utilizado}

Se utiliza un algoritmo genético (AG) tradicional \citep{Goldberg1989} provisto por la biblioteca de optimización Malva \citep{Malva}, implementada en C++. El código del AG fue modificado para incluir las características del problema a resolver y para proveer la funcionalidad de ejecución en paralelo siguiendo el modelo maestro-esclavo para evaluar
las posibles soluciones del problema, generando un hilo de ejecución por cada simulación. El siguiente esquema describe el funcionamiento del algoritmo utilizado:

\begin{algorithm}[H]
	\caption{Algoritmo Genético de Malva. }
	\label{alg:algoritmo_genetico_malva}
	\begin{algorithmic} [1] 
		{
			\STATE \texttt{t} = 0
			\STATE {Inicializar( P(t))}
			\STATE {Evaluar estructuras en ( P(t))}			
			\WHILE {\text{No terminar}}
			\STATE \texttt{t}++		
			\STATE {Seleccionar C(t) de P(t-1)}	
			\STATE {Recombinar estructuras en C(t) formando C'(t)}				
			\STATE {Mutar estructuras en C'(t) formando C''(t)}		
			\STATE {Evaluar estructuras en C''(t) generando un hilo de ejecucion por cada una}					
			\STATE {Consolidar valores de la evaluacion}								
			\STATE {Reemplazar P(t) de C''(t) y P(t-1)}								
			\ENDWHILE
		}
	\end{algorithmic}
	
\end{algorithm}

A continuación se realiza un resumen de las características del algoritmo implementado, que en la siguiente sección serán descritas en detalle:
\begin{itemize}
	
	\item Se aplica un modelo de AE paralelo, utilizando el esquema maestro-esclavo, donde en cada iteración el maestro genera un hilo por cada evaluación de la función de \emph{fitness} y luego espera a la terminación de todos los hilos para consolidar los datos. 
	\item Se considera una formulación multiobjetivo del problema, que intenta optimizar tanto la velocidad promedio de vehículos como la de ómnibus, teniendo cada componente un peso específico en la combinación lineal utilizada para definir la función de \emph{fitness}.
	\item La representación de soluciones: es un vector de números naturales, que representan la duración de las fases de los semáforos y el \emph{offset} para todos los semáforos del Corredor Garzón.
	\item Respecto a los operadores evolutivos: se implementa una variante del cruzamiento de un punto específico para este problema y se utilizan dos tipos de mutación para modificar la duración de fase y el \emph{offset} de los semáforos.
	\item Respecto a los criterios de selección y reemplazo, el AE propuesto utiliza un modelo generacional y reemplaza padres por hijos. La selección de los padres se realiza por el método de torneo de tres individuos y la selección de hijos por el método de selección proporcional.
	
\end{itemize}

\subsection{Representación de individuos}

%Como se explicó en el marco teórico, el problema de sincronización de semáforos puede ser resuelto optimizando diferentes parámetros. Entre estos parámetros se encuentran la duración de fase, de ciclo y el \emph{offset} de los semáforos. Los parámetros que se seleccionen deberán ser incluidos en la representación del cromosoma.

La solución propuesta contempla dos elementos clave de la planificación de semáforos: la \textit{duración de fase} y el \emph{offset} (en segundos). La Fig.~\ref{fig:cromosoma1} presenta un ejemplo de la representación utilizada para las soluciones del problema: la información se agrupa lógicamente en cruces, el valor de cada \textit{gen} es la duración en segundos de una fase en un cruce (se omiten las fases de  luces amarillas, que no modifican los tiempos reales de paso de vehículos) y el \emph{offset} se encuentra al final de cada cruce representado. El largo de una representación depende de la cantidad de cruces y de la cantidad de fases que incluye. Con la representación propuesta se busca una optimización global de todo el sistema y no individualmente para cada cruce, lo cual es fundamental para mejorar la velocidad promedio del Corredor Garzón y del resto de las calles que lo cruzan.

\begin{figure}[!ht]
	\centering
	\includegraphics[width=0.85\linewidth]{Figures/mapa_genoma}
	\caption{Ejemplo de representación de las soluciones para el problema.} 
	\label{fig:cromosoma1}
\end{figure}

Toda la información del cromosoma se guarda y carga en el formato \emph{XML} del archivo de configuración de semáforos, que utiliza el simulador SUMO. La Figura \ref{fig:rep_sumo} muestra como se modela la esquina de Garzón y Emancipación en el formato antes mencionado. Para las distintas fases de los semáforos de una determinada esquina se utiliza la etiqueta \emph{phase}, donde \emph{state} contiene información sobre la secuencia de luces. Por ejemplo, la fase \emph{GGGGrrrGGGGrrr} indica la secuencia de luces, \emph{G} indica la luz verde con preferencia, \emph{g} indica la luz verde sin preferencia, \emph{r} la luz roja, \emph{y} luz amarilla. La etiqueta \emph{duration} especifica la duración de la fase y corresponde a los valores del cromosoma. El tiempo indicado en la etiqueta \emph{offset} especifica el inicio de la fase y será el correspondiente al valor del gen de \emph{offset} en el cromosoma para el cruce de Garzón y Emancipación.


\begin{figure}[ht]
	\centering
	\includegraphics[width=\linewidth]{Figures/rep_sumo}
	\caption{Representación de la configuración de semáforos en Sumo}
	\label{fig:rep_sumo}
\end{figure}

\newpage

El simulador representa en las esquinas (mediante el uso de flechas blancas) la elección de carriles habilitados para los vehículos que transitan por un determinado carril; en el caso de un giro, la flecha apunta hacia donde se encuentra el carril del giro. La Figura \ref{fig:sem_numerados} muestra como se modela la esquina de Garzón y Emancipación en el simulador, a la que se le añadieron los números del 1 al 14 para indicar todas las flechas blancas de la esquina y establecer la correspondencia con la representación en SUMO en la Figura \ref{fig:rep_sumo}. Por lo tanto, para el estado \emph{GGGGrrrGGGGrrr} se tiene que: \textit{i)} \emph{GGGG} se corresponde con las flechas 1, 2, 3 y 4, \textit{ii)} \emph{rrr} se corresponde con 5, 6 y 7, \textit{iii)} \emph{GGGG} se corresponde con 8, 9, 10 y 11 y \textit{iv)}\emph{rrr} se corresponde a 12, 13 y 14. Estableciendo esta correspondencia para cada cruce, se modelan las secuencias y duraciones de luces originales de manera que represente lo relevado in-situ.

\begin{figure}[H]
	\centering
	\includegraphics[width=0.7\linewidth]{Figures/semaforos_numerado}
	\caption[Representación de una fase de los semáforos para un cruce.]{Representación de una fase de los semáforos para el cruce de Emancipación y Garzón.}
	\label{fig:sem_numerados}
\end{figure}

\subsection{Generación de la población inicial}

Para inicializar la población se toma como referencia la configuración obtenida a partir de los datos de la realidad, recolectadas en el relevamiento realizado. Luego, para cada cruce se varían las duraciones de las fases de manera aleatoria entre un rango de valores configurable. Además, la fase inicial se selecciona aleatoriamente entre la cantidad de fases del cruce.

\subsection{Función de \emph{fitness}}

La función de \emph{fitness} utilizada corresponde a la formulación presentada en el capítulo 4. Se considera una formulación multiobjetivo del problema, que intenta optimizar tanto la velocidad promedio de vehículos como la de ómnibus, teniendo cada componente un peso específico en la combinación lineal utilizada para definir la función de \emph{fitness}. 

\subsection{Operadores evolutivos}

\subsubsection{Operador de cruzamiento}
Se utiliza cruzamiento de un punto. Se selecciona hasta un determinado cruce que se decide aleatoriamente como punto de corte, y se producen dos hijos seleccionando las soluciones de un individuo, desde el primer cruce hasta el punto de corte y se une con las soluciones del otro individuo desde el punto de corte en adelante; el otro hijo se forma de manera inversa. En el escenario que plantea la Figura \ref{fig:op_cruzamiento}, los padres cuentan con tres cruces. Se elige un punto de corte aleatoriamente entre dos cruces, cortando a los padres e intercambiando los trozos del cromosoma para generar los hijos como se ve en la Figura \ref{fig:op_cruzamiento} . 

\begin{figure}[H]
	\centering
	\includegraphics[width=0.8\linewidth]{Figures/alg_cruzamiento}
	\caption{Ejemplo del operador de cruzamiento utilizado}
	\label{fig:op_cruzamiento}
\end{figure}



\subsubsection{Operador de mutación}
Se implementaron dos tipos de mutación:
\begin{itemize}
	
	\item Mutación de duración de fase: para cada fase de cada cruce se modifica su duración sumando o restando una cantidad dada de segundos. Se valida que el nuevo valor obtenido se encuentre dentro de un rango especificado de entre 5 y 60 segundos (valores que se pueden modificar), para que no se produzcan casos alejados de la realidad, por ejemplo una intersección donde los vehículos tengan menos de 5 segundos para cruzar.
	
	\item Mutación del \emph{offset}: Para cada \emph{offset} del cromosoma se suma o resta una cantidad de segundos dada y se valida que esté dentro de un rango prefijado de entre 0 y el máximo valor de \emph{offset} para ese cruce.
	

\end{itemize}
Ambas mutaciones son aplicadas de acuerdo a una tasa de probabilidad prefijada.

\subsubsection{Selección y reemplazo}
Se utilizan los operadores y criterios provistos por el AG de Malva. La selección de padres se realiza por el método de torneo de tres individuos y la selección de hijos por el método de selección proporcional. Respecto a la política de reemplazo, se utiliza un modelo ``+", en el cual los hijos compiten con los padres por la supervivencia en la siguiente generación.

\subsubsection{Parámetros del algoritmo}
Los parámetros del algoritmo son: el número de generaciones que se realizan hasta detener el algoritmo, el tamaño de la población que indica la cantidad de individuos incluidos en el AE, la tasa de probabilidad de cruzamiento y la de mutación que se utiliza para determinar cuando aplicar los operadores de cruzamiento y de mutación. Los valores específicos de los parámetros del AE se ajustan en el capítulo 5 donde se realiza el análisis de configuración paramétrica. 

\section{Modelo de paralelismo e implementación}


%The skeletons in MALLBA offer support for parallelism using the distributed memory approach (i.e., implementing distributed subpopulation models for metaheuristics). However, the library does not provide support for shared-memory multithreading parallel programming. Multihtreading programming allows implementing efficient algorithms by using multiple threads within a single process. Multihtreading is well suited for multi-core computers, where each thread is executed on a single core. It provides a fast method for concurrent execution; communications and synchronizations are performed via the shared-memory resource, which is handled using mutuallyexclusive operations in order to prevent simultaneous accesses.

%The multithreading master-slave parallel EA proposed in this work was implemented using the GA skeleton in MALLBA. Additional code was incorporated into the GA skeleton to implement several new features:
%– to create and manage the pool of threads used for the fitness evaluation;
%– to implement the master-slave hierarchy and the communications between master and slaves;
%– to define the synchronization mechanisms between threads,
%used to read and write the shared memory. Our implementation starts by creating and initializing a pool of threads to distribute the fitness evaluation. Each thread receives several input parameters from the master process, including the solution to be evaluated, the thread identification, and the index in the array of fitness values.Then, each slave process, implemented in each thread, computes the fitness evaluation by simulating the mobile communications with the proposed


Uno de los requisitos planteados al inicio del presente trabajo era la creación de un AG que soportara paralelismo, con el objetivo de reducir los tiempos de ejecución de las simulaciones necesarias para evaluar la función de \emph{fitness} de los individuos. El motivo principal fue que los escenarios planteados se consideraron complejos por lo que la ejecución del AE insumiría un tiempo considerable. El código base del AE fue modificado para que soportara la ejecución en paralelo. Específicamente, se desarrolló una nueva forma de evaluar a los individuos, generando un hilo de ejecución por cada individuo.

El código del AG provisto por la biblioteca Malva (newGA.pro.hh) fue modificado agregando dos nuevas funciones encargadas de llamar a la función de \emph{fitness} y asignar el valor obtenido al individuo tanto para padres (\emph{evaluate{\_}parent{\_}threads}) como para hijos (\emph{evaluate{\_}offpring{\_}threads}). En la Figura \ref{fig:codigo1} se presenta el código de la función \emph{evaluate{\_}parent{\_}threads} que se explica a continuación. 

\begin{figure}[H]
	\centering
	\includegraphics[width=0.99\linewidth]{Figures/codigo1}
	\caption{Función para la evaluación de individuos en paralelo utilizando threads, agregada al código provisto por Malva.}
	\label{fig:codigo1}
\end{figure}

En el código de la Figura \ref{fig:codigo1} el parámetro \emph{data} contiene información sobre la población, el individuo a evaluar, su índice y un vector con los valores de \emph{fitness} de la población. En la línea 378 se hace un chequeo para saber si el valor de \emph{fitness} ya fue obtenido para el individuo, de esta forma no se repiten evaluaciones, en la línea 380 se ejecuta la función de \emph{fitness} y se aplica el valor de \emph{fitness} al individuo, luego en la línea 381 se incrementa el número de evaluaciones realizadas en la población.
La función \emph{evaluate{\_}offpring{\_}threads} es análoga a \emph{evaluate{\_}parent{\_}threads} pero específica para los hijos, por lo que no se considera necesario mostrar su código.
	
La función \emph{evaluate{\_}parents()} provista por Malva se encarga de iterar sobre los padres de la población y ejecutar la evaluación sobre cada uno de ellos para obtener el valor de \emph{fitness} asociado. Esta función fue modificada para generar un hilo de ejecución por cada evaluación, como se aprecia en la porción de código de la Figura \ref{fig:codigo2}.
En la línea 434  se genera el hilo llamando a la función agregada anteriormente \emph{evaluate{\_}parent{\_}threads}. Luego en la línea 504 se espera a que todos los hilos finalicen su ejecución. Análogamente se realizan modificaciones similares en la función 	\emph{evaluate{\_}offsprings }que itera sobre los hijos, para que genere un hilo por cada evaluación de los hijos, llamando a la función de  \emph{evaluate{\_}offpring{\_}threads}.
		
\begin{figure}[H]
	\centering
	\includegraphics[width=1\linewidth]{Figures/codigo2}
	\caption{Modificaciones de la función evaluate{\_}parents.}
	\label{fig:codigo2}
\end{figure}		

Se utilizó el esquema maestro-esclavo para el modelo de paralelismo. El proceso maestro se encarga de la mayoría de las etapas del AE, al comenzar, inicializa la población y se encarga de distribuir la evaluación de los individuos hacia los esclavos, creando un hilo de ejecución por cada esclavo. Luego el proceso maestro espera a que las evaluaciones terminen para obtener los valores de \emph{fitness} de los esclavos. Una vez obtenidos los valores, selecciona a los mejores individuos y le aplica los operadores de cruzamiento y mutación. Para finalizar ejecuta el criterio de reemplazo para generar la siguiente población. Mientras, los esclavos se encargan solamente de obtener el individuo enviado por el maestro y de efectuar la evaluación del individuo, que corresponde a ejecutar el simulador y obtener los datos de salida. 


El modelo de paralelismo no cambia el comportamiento del algoritmo secuencial, sólo se enfoca en mejorar la eficiencia computacional para poder abordar escenarios realistas y complejos.

En el capítulo siguiente, en la sección de análisis experimental se realiza un estudio de la eficiencia computacional del AG paralelo, para comparar los tiempos de ejecución entre la versión paralela y la secuencial.