{
\thispagestyle{empty}
~\\[0.2cm]
\begin{center}
    \textsc{\huge Algoritmos evolutivos en  } \\[0.2cm] 
    \textsc{\huge sincronización de semáforos en el  } \\[0.2cm]         
    \textsc{\huge Corredor de Garzón} \\[1cm]
    \textsc{\Large Resumen}
\end{center}
~\\[0.2cm]
\textbf{\large 
El proyecto propone el estudio de la sincronización de semáforos como problema de optimización multiobjetivo, y el diseño e implementación de un algoritmo evolutivo para resolver el problema con alta eficacia numérica y desempeño computacional en un escenario real utilizando simulaciones del tráfico. \newline \newline
Se toma como aplicación la sincronización de semáforos del “Corredor de Garzón” (Montevideo, Uruguay) dado que la cantidad de cruces, calles, trafico y cantidad de semáforos lo hace un problema interesante desde el punto de vista de su complejidad.Ademas es real y admitido por las autoridades responsables de que hubo problemas en este sentido por lo que todavía hay espacio para la mejora de los tiempos promedio de los viajes.  \newline \newline
El problema de sincronización de semáforos es NP-difícil y no existe (hasta el momento) un método determinístico que lo resuelva, se buscará mediante un algoritmo evolutivo llegar a una configuración aceptable de los semáforos para un conjunto de escenarios, minimizando los tiempos de espera de los vehículos y mejorando de esta manera la calidad del trafico.
Buscamos demostrar que estas técnicas son aplicables a escenarios reales y que deberían considerarse en el abanico de posibilidades disponibles. } 	
	~\\[1.0cm]
    \textbf{\large Palabras clave: Algoritmo Evolutivo, Sincronizacion semáforos, escenario real, Cluster}

}
\cleardoublepage
