{
    \thispagestyle{empty}
    ~\\[0.2cm]
    \begin{center}
        \textsc{\huge Algoritmos evolutivos en  } \\[0.2cm] 
        \textsc{\huge sincronización de semaforos en el  } \\[0.2cm]         
        \textsc{\huge Corredor de Garzón} \\[1cm]
        \textsc{\Large Resumen}
    \end{center}
    ~\\[0.2cm]
    \textbf{\large Este trabajo estudia la aplicación de un algoritmo evolutivo multiobjetivo  la resolucion de problemas de sincronizacion de semaforos en un escenario real. Este tipo de problemas son NP-HARD es decir que no existe un algoritmo que lo resuelva en tiempo polinomico. Los algoritmo evolutivos han sido aplicados con éxito a este tipo de problemas.   
   	Luego  de  una  inversión  que  se  asume  fue  de
   	aproximadamente  33  millones  de  dólares  y  una  infraestructura
   	que  pretendía  ser  de  primer  mundo,  no  se  pudo  agilizar  el
   	transporte público lo cual era el único objetivo que tuvo desde un
   	principio en la obra llamada “Corredor de Garzón”. Se pretende
   	en  este  trabajo  mejorar  mediante  algoritmos  evolutivos  la
   	sincronización  de  los  semáforos,  para  poder  lograr  agilizar  el
   	transporte, minimizando los tiempos de espera.
   	Presentaremos  los  resultados  y  tiempos  de  ejecución  del
   	algoritmo y compararemos con los datos recabados inicialmente,
   	con  los  obtenidos  en  el  algoritmo  utilizando  el  simulador  de
   	tráfico  SUMO  en  distintos  escenarios} 	
	~\\[1.0cm]
    \textbf{\large Palabras clave: Algoritmo Evolutivo, Sincronizacion semaforos, escenario real, Cluster}

}
\cleardoublepage
